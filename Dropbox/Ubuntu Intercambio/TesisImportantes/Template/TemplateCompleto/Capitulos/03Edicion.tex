%---------------------------------------------------------------------
%
%                          Capítulo 3
%
%---------------------------------------------------------------------
%
% 03Edicion.tex
% Copyright 2009 Marco Antonio Gomez-Martin, Pedro Pablo Gomez-Martin
%
% This file belongs to the TeXiS manual, a LaTeX template for writting
% Thesis and other documents. The complete last TeXiS package can
% be obtained from http://gaia.fdi.ucm.es/projects/texis/
%
% Although the TeXiS template itself is distributed under the 
% conditions of the LaTeX Project Public License
% (http://www.latex-project.org/lppl.txt), the manual content
% uses the CC-BY-SA license that stays that you are free:
%
%    - to share & to copy, distribute and transmit the work
%    - to remix and to adapt the work
%
% under the following conditions:
%
%    - Attribution: you must attribute the work in the manner
%      specified by the author or licensor (but not in any way that
%      suggests that they endorse you or your use of the work).
%    - Share Alike: if you alter, transform, or build upon this
%      work, you may distribute the resulting work only under the
%      same, similar or a compatible license.
%
% The complete license is available in
% http://creativecommons.org/licenses/by-sa/3.0/legalcode
%
%---------------------------------------------------------------------

\chapter{Algoritmo del Método del Realizador de Schnyder}
\label{cap3}
\label{cap:edicion}

%\begin{FraseCelebre}
%\begin{Frase}
%Si quieres ser leído más de una vez, no vaciles en borrar a menudo.
%Rem tene, verba sequentur (Si dominas el tema, las palabras vendrán solas)
%\end{Frase}
%\begin{Fuente}
%Horacio
%Catón el Viejo
%\end{Fuente}
%\end{FraseCelebre}

\begin{resumen}
  Este capítulo se describirá de manera detallada el Método Realizador (\emph {Realizer Method}), éste fue creado por Schnyder y sirve para encontrar el dibujo de una gráfica plana maximal en una malla (\emph{grid}) de tamaño $(n-2)\times (n-2) $, utilizando segmentos de línea recta y coordenadas enteras.
\end{resumen}

%-------------------------------------------------------------------
\section{Empezando a escribir}
%-------------------------------------------------------------------
\label{cap3:sec:tituloYmetadatas}

En primer lugar, es necesario destacar que los ficheros \texttt{.tex}
\emph{deben tener} codificación ISO-8859-1. Esto es lo que ocurre de
manera predefinida en Windows y en algunos Linux como Debian. Una
excepción significativa es el caso de Ubuntu, que usa de manera
predeterminada UTF-8. En ese caso, deberás ser cuidadoso para
asegurarte de que grabas tus ficheros con ISO-8859-1.

\com{En realidad, hay una remota posibilidad de que TeXiS se pueda
  configurar para usar UTF-8 de manera nativa, aunque \emph{nunca lo
    hemos probado} y \emph{no} lo recomendamos.  Hay información
  adicional en un comentario en \texttt{TeXiS/TeXiS\_pream.tex} (busca
  \texttt{inputenc} para encontrarlo).}

\medskip

El primer paso para la construcción de un nuevo documento es cambiar
el título y autores. Es posible que al principio del proceso no se
tenga muy claro cuál es el título final del documento pero, y esto es
una opinión personal, ver un título (aunque sea provisional) en vez de
lo que ahora aparece (``\titulo'') te ayudará a pensar que lo que
estás escribiendo es tuyo y no de otros. Para eso, basta con cambiar
la constante \verb|\titulo| y \verb|\autor| que aparece definida en el
fichero \texttt{constantes.tex}.

El segundo paso es crear la portada en
\texttt{Cascaras/cover.tex}. Como habrás podido observar, \texis\
genera dos hojas de portada, al igual que hacen la mayoría de los
libros. La primera portada es la que iría en la parte exterior del
documento encuadernado, mientras que la siguiente es una repetición
que aparece en la primera página. A continuación aparece una lista con
el texto que puede cambiarse usando los comandos de \texis; una vez
que se configuran, se debe invocar al comando \verb|\makeCover| para
generar las portadas:

\begin{itemize}
\item Título del documento: aparece en las dos portadas. Por defecto
  se utilizará la constante \verb|\titulo| definida en
  \texttt{constantes.tex}. No obstante, se puede indicar un título
  distinto usando \verb|\tituloPortada|. De esta forma, se pueden
  forzar saltos de línea artificiales si se desea.

\item Autor del documento: normalmente aparece también en las dos
  portadas. Igual que antes, si no se indica lo contrario se utiliza
  \verb|\autor|, aunque se puede cambiar con \verb|\autorPortada|.

\item Una imagen en la primera portada, normalmente el escudo
  institucional. El fichero a utilizar se define con
  \verb|\imagenPortada|. También puede especificarse la escala a
  utilizar en el fichero si éste es demasiado grande o pequeño con
  \verb|\escalaImagenPortada|.

\item Una fecha de publicación, que aparece en la parte inferior de
  ambas portadas. Se utiliza el comando \verb|\fechaPublicacion|.

\item El ``tipo de documento'' que aparece en la primera portada. Si
  no se indica nada, será ``TESIS DOCTORAL''. Se puede modificar con
  \verb|\tipoDocumento|. Este manual por ejemplo lo establece en
  ``MANUAL DE USUARIO''.

\item El departamento y facultad al que está asociado el
  documento. Aparece en ambas portadas, y se establece con
  \verb|\institucion|.

\item Un primer bloque de texto en la segunda portada, que aparece
  después del título. Si no se indica lo contrario, en ese bloque
  aparecerá el texto ``Memoria que presenta para optar al título de
  Doctor en Informática'' seguido del \verb|\autorPortada|. Se puede
  cambiar el contenido completo con
  \verb|\textoPrimerSubtituloPortada|.

\item Un segundo bloque de texto donde aparece ``Dirigida por el
  Doctor'' seguido del director del trabajo que se establece con
  \verb|\directorPortada|. El comando
  \verb|\textoSegundoSubtituloPortada| permite establecer otro texto
  distinto.
\end{itemize}

Las dos portadas en sus caras traseras pueden, además, presentar otra
información auxiliar:

\begin{itemize}
\item Un breve recordatorio indicando que el documento está preparado
  para su impresión a doble cara. Si se desea que aparezca, basta con
  llamar a \verb|\explicacionDobleCara|.

\item El ISBN del documento, en caso de poseerlo. Se define con
  \verb|\isbn|.

\item Información de copyright. Se puede indicar con
  \verb|\copyrightInfo|, y lo habitual será pasar como parámetro el
  \verb|\autor|.

\item Por defecto en la cara posterior de la primera portada aparecen
  unos ``créditos'' a \texis, donde se indica que el documento se ha
  generado con \texis\ y la versión. Si no se desea que aparezca, se
  puede llamar a \verb|\noTeXiSCredits|, aunque nos gustaría que lo
  incluyeras.
\end{itemize}

Por último, quizá quieras cambiar la información de ``metadatos'' que
se incrustará en el PDF generado. Los metadatos aparecen directamente
en el fichero \texttt{Tesis.tex} y, como indicamos en el capítulo
anterior y mostramos en la figura~\ref{cap2:fig:pdf}, son:

\begin{verbatim}
%
% "Metadatos" para el PDF
%
\ifpdf\hypersetup{%
    pdftitle = {\titulo},
    pdfsubject = {Plantilla de Tesis},
    pdfkeywords = {Plantilla, LaTeX, tesis, trabajo de
      investigación, trabajo de Master},
    pdfauthor = {\textcopyright\ \autor},
    pdfcreator = {\LaTeX\ con el paquete \flqq hyperref\frqq},
    pdfproducer = {pdfeTeX-0.\the\pdftexversion\pdftexrevision},
    }
    \pdfinfo{/CreationDate (\today)}
\fi
\end{verbatim}

%Para adecuarlo a tu documento concreto, deberás cambiar la entrada del
%tema y palabras clave. El título y autores se rellenan con las
%contantes correspondientes.

%-------------------------------------------------------------------
\section{Editando el texto}
%-------------------------------------------------------------------
\label{cap3:sec:edicion}

Una vez que se tiene el título y autores del documento puestos, el
trabajo de escritura consiste, en su mayor parte, en la creación de
los correpondientes ficheros \LaTeX\ de cada uno de los capítulos y
apéndices.

%-------------------------------------------------------------------
\subsection{Nuevos capítulos (y apéndices)}
%-------------------------------------------------------------------

Según la estructura de directorios vista en el capítulo anterior,
\texis\ te recomienda crear los capítulos en el directorio
\texttt{Capitulos} y los apéndices en \texttt{Apendices}.

Cuando crees un fichero en cualquiera de los directorios, se debe
añadir en el fichero maestro (\texttt{Tesis.tex}) el nombre de ese
nuevo fichero para que se procese en el momento de la generación:

\begin{verbatim}
\mainmatter

%---------------------------------------------------------------------
%
%                          Cap�tulo 1
%
%---------------------------------------------------------------------
%
% 01Introduccion.tex
% Copyright 2009 Marco Antonio Gomez-Martin, Pedro Pablo Gomez-Martin
%
% This file belongs to the TeXiS manual, a LaTeX template for writting
% Thesis and other documents. The complete last TeXiS package can
% be obtained from http://gaia.fdi.ucm.es/projects/texis/
%
% Although the TeXiS template itself is distributed under the 
% conditions of the LaTeX Project Public License
% (http://www.latex-project.org/lppl.txt), the manual content
% uses the CC-BY-SA license that stays that you are free:
%
%    - to share & to copy, distribute and transmit the work
%    - to remix and to adapt the work
%
% under the following conditions:
%
%    - Attribution: you must attribute the work in the manner
%      specified by the author or licensor (but not in any way that
%      suggests that they endorse you or your use of the work).
%    - Share Alike: if you alter, transform, or build upon this
%      work, you may distribute the resulting work only under the
%      same, similar or a compatible license.
%
% The complete license is available in
% http://creativecommons.org/licenses/by-sa/3.0/legalcode
%
%---------------------------------------------------------------------

\chapter{Introducci�n}

\begin{FraseCelebre}
\begin{Frase}
P�sose don Quijote delante de dicho carro, y haciendo en su fantas�a
uno de los m�s desvariados discursos que jam�s hab�a hecho, dijo en
alta voz:
\end{Frase}
\begin{Fuente}
  Alonso Fern�ndez de Avellaneda, El Ingenioso Hidalgo Don Quijote de
  la Mancha
\end{Fuente}
\end{FraseCelebre}

\begin{resumen}
  Este cap�tulo presenta una breve introducci�n a \texis.  El
  lector podr� hacerse una idea de qu� es y para qu� sirve. Tambi�n se
  encuentra aqu� una descripci�n del resto de cap�tulos del manual.
\end{resumen}


%-------------------------------------------------------------------
\section{Introducci�n}
%-------------------------------------------------------------------
\label{cap1:sec:introduccion}


Si est�s leyendo estas l�neas es muy posible que haya llegado la hora
de ponerte a escribir la tesis, despu�s de mucho tiempo dando vueltas
al �rea de investigaci�n concreta en el que est�s inmerso. O puede que
est�s a punto de empezar a escribir la memoria del proyecto de fin de
carrera, fin de master, o cualquier otro documento de cierta
envergadura.

Sea lo que sea lo que te traes entre manos, lo m�s probable es que no
sea f�cil hacerlo. Muy posiblemente no tengas a�n muy claro qu� vas a
escribir, pero tu tutor/director/profesor te ha dicho que vayas
empezando a plasmar esas ideas sobre el papel para tener algo firme, y
sentir que vas avanzando.

Y entonces viene el problema de c�mo escribirlo. Muy posiblemente
habr�s escrito alg�n art�culo en \LaTeX\ y est�s convencido de que esa
es la v�a a seguir para hacer un documento que superar� las 10 p�ginas
y que tendr� bibliograf�a. O puede, simplemente, que alguien te haya
dicho que lo mejor es que escribas el proyecto en \LaTeX\ porque la
apariencia final es mejor, porque es m�s c�modo, o cualquier otra
raz�n.

Sea como fuere, parece que est�s m�s o menos decidido a escribir tu
documento en \LaTeX. Bien hecho. Pero, �c�mo?. Al contrario de lo que
suele ocurrir en congresos y en revistas, no tienes disponible ninguna
p�gina en la que descargarte las ``instrucciones para los autores'',
con la c�moda plantilla en \LaTeX\ que t�, sufrido autor, simplemente
tienes que rellenar. No. Ahora las cosas son m�s complicadas.

As� que te vas a la gu�a de \LaTeX\ con la que empezaste (apostamos
que es la misma con la que hemos empezado todos), y ves las distintas
posibilidades que te ofrece en su ``\texttt{documentclass}'':
\texttt{article}, \texttt{report}, \texttt{book}, ... Y te quedas con
la �ltima. Pero te asaltan muchas preguntas. �C�mo organizo todo esto?
o �c�mo hago la portada? o incluso �qu� hago para que no ponga
``Chapter'', sino ``Cap�tulo''?. En ese punto, es de suponer, has
pedido ayuda a la gente de alrededor y/o a tu buscador de Internet
favorito. Y de alguna forma, te has encontrado leyendo estas l�neas.

Tenemos que decir que exactamente esa fue nuestra situaci�n cuando por
fin nos decidimos a escribir nuestras tesis. Desgraciadamente, ni la
gente que ten�amos alrededor ni nuestro buscador favorito supieron
contestarnos de forma satisfactoria, por lo que tuvimos que invertir
\emph{mucho tiempo} hasta conseguir que el resultado que sal�a de
nuestros \texttt{.tex} nos gustara, hasta que nos sentimos c�modos con
la estructura de los ficheros, con las macros disponibles y con el
modo de compilaci�n.

Y para que nadie m�s pueda utilizar como excusa el no saber c�mo
personalizar la clase \texttt{book} para retrasar el comienzo de su
tesis, para que nadie m�s se decida por Word u otro paquete ofim�tico
en vez de \LaTeX\ porque lo ve mucho m�s sencillo, en definitiva, para
que nadie pierda tanto tiempo como perdimos nosotros creando la
estructura, decidimos hacer p�blico el esqueleto b�sico que
construimos nosotros para hacerlas. Ese esqueleto b�sico o plantilla
es \texis.

En vez de hacer disponible la plantilla o ficheros \texttt{.tex} sin
ning�n contenido, proporcionamos un manual en formato PDF que (a no
ser que est�s leyendo directamente el c�digo \LaTeX), ser� lo que
est�s leyendo. Este manual ha sido creado \emph{con la propia
  plantilla}. Por lo tanto, la distribuci�n de \texis\ es en
realidad el c�digo fuente de \emph{su propio manual}. Con su c�digo
fuente entre tus manos, lo �nico que tienes que hacer es borrar su
contenido (\emph{este texto}), y rellenarlo con tu gran contribuci�n
al mundo.  Como podr�s comprobar, la estructura del propio manual
sigue el esquema de lo que podr�a ser una tesis, trabajo de
investigaci�n o proyecto de fin de carrera, precisamente para que sea
f�cil quitar el contenido textual y sustituirlo por el nuevo.

En los cap�tulos que siguen encontrar�s toda la informaci�n necesaria
para poder utilizar los ficheros \LaTeX\ para crear tus propios
documentos. Adem�s, el propio c�digo fuente est� lleno de comentarios
(especialmente en los ficheros que definen el estilo), por lo que
tambi�n en ellos encontrar�s una buena fuente de informaci�n. Eso es
especialmente importante en caso de que quieras modificar en algo el
aspecto final de tu documento.

Esperemos que te sea de utilidad. Si es as�, nos gustar�a que lo
reconocieras en la secci�n de agradecimientos. Si durante tu proceso
de escritura has a�adido alg�n aspecto que crees que puede ser
interesante para otros, no dudes en dec�rnoslo para intentar incluirlo
en siguientes versiones de la propia plantilla; tampoco dudes en
enviarnos sugerencias sobre las explicaciones de este manual para
poder mejorarlo con el tiempo. Por �ltimo, tambi�n puedes enviarnos el
resultado final para poner una referencia a �l en la p�gina de
descarga, donde, por cierto, puedes ver otros documentos creados con
la plantilla, lo que te permitir� coger ideas de cosas que puedes
variar. Recuerda que la versi�n m�s reciente de \texis\ est�
disponible en \url{http://gaia.fdi.ucm.es/projects/texis/}.

%-------------------------------------------------------------------
\section{Qu� es \texis}
%-------------------------------------------------------------------
\label{cap1:sec:que-es}

La plantilla que tienes entre las manos es, como hemos dicho, el
esqueleto del c�digo fuente de las Tesis Doctorales de los dos autores
\citep{GomezMartinMA2008PhD, GomezMartinPP2008PhD}. Por tanto, sirve
para escribir otras Tesis Doctorales u otros documentos con estructura
similar de forma f�cil.

\texis\ te permite adem�s generar el fichero utilizando tanto el
comando \texttt{latex} (que genera de forma nativa ficheros
\texttt{dvi} que luego se convierten a ficheros \texttt{ps} o
\texttt{pdf}), como \texttt{pdflatex}. De esta forma el usuario final
puede elegir entre cualquiera de las dos herramientas\footnote{Esto es
  �til por ejemplo cuando quieres utilizar \texttt{pdflatex} pero
  finalmente el servicio de publicaciones s�lo admite el uso de
  \texttt{latex}.}.  Aconsejamos, no obstante, la utilizaci�n de este
�ltimo, debido a que \texis\ contiene ciertos comandos para dotar al
PDF final de marcadores que permiten una navegaci�n c�moda por el
fichero utilizando los visores tradicionales.

\medskip

Como explicaremos en el cap�tulo siguiente, la plantilla se aprovecha
mejor en sistemas GNU/Linux. Nota que hemos dicho que la plantilla
``\emph{se aprovecha mejor}'' en sistemas GNU/Linux, no que \emph{no
pueda utilizarse} en Windows o Mac; es evidente que \LaTeX\ es
multiplataforma, y por lo tanto puede compilarse en cualquier sistema
que tenga instalada una distribuci�n del mismo.

La raz�n por esta ``desviaci�n positiva'' hacia Linux estriba en que
para hacer m�s c�modo el proceso de edici�n y compilaci�n, \texis\
proporciona ficheros que facilitan el proceso de generaci�n del
fichero PDF final, tal y como se describe en el
cap�tulo~\ref{cap:makefile}.  Esos ficheros adicionales s�lo funcionan
correctamente si son ejecutados en Linux.

%-------------------------------------------------------------------
\section{Qu� no es}
%-------------------------------------------------------------------
\label{cap1:sec:que-no-es}

Esta plantilla \emph{no} es un manual de \LaTeX, ni una gu�a de
referencia, ni un compendio de preguntas frecuentes. De hecho, no nos
consideramos expertos en \LaTeX, por lo que no tendr�amos fuerzas para
escribir algo as�. Si necesitas un manual de \LaTeX, puedes encontrar
muchos y muy buenos en Internet. Al final de este cap�tulo aparece una
lista con algunos de ellos.

La plantilla tampoco es \emph{una clase} de \LaTeX. Si miras el c�digo
fuente podr�s comprobar que el documento comienza con
\verb+\documentclass{book}+\footnote{Personalizado, eso s�, para que
  utilice DIN A-4, a doble cara y con letra de 11 puntos.}, por lo que
se basa en la clase \texttt{book}.

La plantilla tampoco te ayudar� a gestionar tu bibliograf�a. Los
\texttt{.bib} los tendr�s que crear y organizar t� ya sea de forma
manual o con alguna herramienta dise�ada para ello.

\medskip

Queremos una vez m�s insistir antes de terminar que no somos expertos
en \LaTeX.  Durante el proceso de escritura de nuestras Tesis nos
tuvimos que enfrentar a problemas de formato que tuvimos que
solucionar buscando en Internet o preguntando a personas cercanas. Y
podemos decir que pr�cticamente todos los problemas a los que nos
hemos enfrentado en nuestra vida como usuarios de \LaTeX\ est�n
resueltos aqu�, pues sendas Tesis han sido los documentos m�s extensos
que hemos escrito.

Por lo tanto, si tienes alguna duda concreta de \LaTeX, en vez de
preguntarnos a nosotros, busca en foros de Internet o en la
documentaci�n del paquete que est�s utilizando. A buen seguro
encontrar�s ah� la respuesta. Si la duda que tienes es relativa a la
plantilla, revisa los comentarios que encontrar�s en el c�digo fuente,
hay ciertas cosas de demasiado bajo nivel que hemos preferido no
contar en el texto. Y s�lo como �ltimo recurso, preguntanos a
nosotros, aunque ya te advertimos que puede que no sepamos
responderte. Querr�amos poder animarte a escribirnos tus dudas, pero
preferimos no hacerlo para no decepcionarte.


%-------------------------------------------------------------------
\section{Estructura de cap�tulos}
%-------------------------------------------------------------------
\label{cap1:sec:estructura}

El manual est� estructurado en los siguientes cap�tulos:

\begin{itemize}
\item El cap�tulo~\ref{cap2} describe a vista de p�jaro los distintos
  ficheros que forman \texis. Adem�s da una primera aproximaci�n
  a c�mo generar el documento final (\texttt{.pdf}).

\item El cap�tulo~\ref{cap3} se centra en el proceso de
  edici�n. Aunque aparentemente la tarea de escribir el texto es
  trivial, \texis\ proporciona una serie de comandos que pueden
  ser �tiles durante la escritura (al menos a nosotros nos lo
  parecieron). Este cap�tulo se centra en la explicaci�n de esos
  comandos.

\item El cap�tulo~\ref{cap4} pasa a describir c�mo se estructuran las
  im�genes en \texis. Igual que antes, esto puede parecer
  superfluo a un usuario medio de \LaTeX, pero \texis\ contiene
  algunos comandos que esperan esa estructura. Es el usuario el �ltimo
  que decide si utiliza esos comandos (y por lo tanto esa estructura)
  u opta por otra completamente distinta.

\item El cap�tulo~\ref{cap5} aborda la bibliograf�a y la gesi�n de los
  acr�nimos. Como se ver�, \texis\ dispone de algunas opciones de
  personalizaci�n que merecen un peque�o cap�tulo.

\item El cap�tulo~\ref{cap6} pone fin al manual, detallando las
  opciones del fichero \texttt{Makefile} que permiten una generaci�n
  c�moda del documento final en entornos Linux.
\end{itemize}

El manual tiene, por �ltimo, un ap�ndice que, si bien no es
interesante desde el punto de vista del usuario, nos sirve de excusa
para proporcionar el c�digo \LaTeX\ necesario para su creaci�n: a modo
de ``as� se hizo'', comenta brevemente c�mo fue el proceso de
escritura de nuestras tesis.


%-------------------------------------------------------------------
\section*{\NotasBibliograficas}
%-------------------------------------------------------------------
\TocNotasBibliograficas

El ``libro'' por el que la mayor�a de la gente empieza sus andaduras
con \LaTeX\ es \cite{ldesc2e} pues es relativamente corto, f�cil de
leer y de acceso p�blico (licencia GPL), por lo que se puede
conseguir la versi�n electr�nica f�cilmente. Un libro algo m�s
completo que �ste y que suele ser el segundo en orden de preferencia
es \cite{notsoshort} con la misma licencia. Dentro de los libros
dedicados a \LaTeX\ de libre distribuci�n, tambi�n se puede contar con
\cite{latexAPrimer}.

No obstante, los libros de \LaTeX\ m�s conocidos son ``The \LaTeX\
Companion'' \citep{latexCompanion} y ``\LaTeX: A Document Preparation
System'' \citep{LaTeXLamport}.

%-------------------------------------------------------------------
\section*{\ProximoCapitulo}
%-------------------------------------------------------------------
\TocProximoCapitulo

Una vez hecha una descripci�n de \texis, el pr�ximo cap�tulo
describe los ficheros que componen tanto la plantilla como el manual
que est�s leyendo. Tambi�n se explicar� c�mo se puede generar o
compilar el manual a partir de los \texttt{.tex} proporcionados. Por
lo tanto, el cap�tulo sirve como una primera aproximaci�n r�pida al
trabajo con \texis; al final del mismo seremos capaces de entender
la estructura de directorios propuesta y d�nde se encuentran los
ficheros que hay que editar para cambiar el contenido del documento
final.

No obstante, el cap�tulo siguiente debe verse �nicamente como una
primera aproximaci�n. El cap�tulo~\ref{cap:edicion} da m�s detalles
sobre el proceso de edici�n del documento, y el
cap�tulo~\ref{cap:makefile} dar� una alternativa al modo de
compilaci�n explicado.

% Variable local para emacs, para  que encuentre el fichero maestro de
% compilaci�n y funcionen mejor algunas teclas r�pidas de AucTeX
%%%
%%% Local Variables:
%%% mode: latex
%%% TeX-master: "../ManualTeXiS.tex"
%%% End:

%---------------------------------------------------------------------
%
%                          Cap�tulo 2
%
%---------------------------------------------------------------------
%
% 02EstructuraYGeneracion.tex
% Copyright 2009 Marco Antonio Gomez-Martin, Pedro Pablo Gomez-Martin
%
% This file belongs to the TeXiS manual, a LaTeX template for writting
% Thesis and other documents. The complete last TeXiS package can
% be obtained from http://gaia.fdi.ucm.es/projects/texis/
%
% Although the TeXiS template itself is distributed under the 
% conditions of the LaTeX Project Public License
% (http://www.latex-project.org/lppl.txt), the manual content
% uses the CC-BY-SA license that stays that you are free:
%
%    - to share & to copy, distribute and transmit the work
%    - to remix and to adapt the work
%
% under the following conditions:
%
%    - Attribution: you must attribute the work in the manner
%      specified by the author or licensor (but not in any way that
%      suggests that they endorse you or your use of the work).
%    - Share Alike: if you alter, transform, or build upon this
%      work, you may distribute the resulting work only under the
%      same, similar or a compatible license.
%
% The complete license is available in
% http://creativecommons.org/licenses/by-sa/3.0/legalcode
%
%---------------------------------------------------------------------

\chapter{Estructura  y generaci�n}
\label{cap2}

\begin{FraseCelebre}
\begin{Frase}
  La mejor estructura no garantizar� los resultados ni el rendimiento.
  Pero la estructura equivocada es una garant�a de fracaso.
\end{Frase}
\begin{Fuente}
Peter Drucker
\end{Fuente}
\end{FraseCelebre}

\begin{resumen}
  Este cap�tulo explica la estructura de directorios de \texis\
  as� como los ficheros m�s importantes, describiendo el cometido de
  cada uno. Tambi�n hace una primera aproximaci�n al proceso de
  generaci�n (o compilaci�n) del PDF final, aunque este tema ser�
  extendido posteriormente en el
  cap�tulo~\ref{cap:makefile}.
\end{resumen}

%-------------------------------------------------------------------
\section{Estructura de directorios}
%-------------------------------------------------------------------
\label{cap2:sec:estructura}

Como habr�s podido comprobar, la plantilla contiene bastantes ficheros
organizados en varios directorios. Esta secci�n explica el contenido
de cada uno de los directorios, para que seas capaz de encontrar el
directorio en el que deber�a estar un fichero concreto.

Existen los siguientes directorios:

\begin{description}
\item[Directorio ra�z] contiene el fichero principal del documento
  (tambi�n llamado fichero \emph{maestro}), que es el que se utiliza
  como entrada a \texttt{pdflatex} (o \texttt{latex}) y cuyo nombre es
  \texttt{Tesis.tex}. Tambi�n aparecen en el directorio otros ficheros
  que si bien no generan texto en el documento final cumplen ciertas
  funciones espec�ficas descritas en la
  secci�n~\ref{cap2:sec:directorio-raiz}. Por �ltimo, el directorio
  contiene tambi�n los ficheros \texttt{.bib} con la informaci�n
  bibliogr�fica as� como el fichero para generar el documento
  utilizando la aplicaci�n \verb+make+.

\item[Directorio \texttt{./Capitulos}] contiene los \texttt{.tex} de
  cada cap�tulo del documento.

\item[Directorio \texttt{./Apendices}] contiene los \texttt{.tex} de
  cada uno de los ap�ndices.

\item[Directorio \texttt{./Cascaras}] contiene los \texttt{.tex}
  responsables del contenido del resto de p�ginas del documento: el
  texto de la portada, agradecimientos, resumen, etc. En definitiva
  son los ficheros responsables de todo aquello que precede a los
  cap�tulos y sigue a los ap�ndices.

\item[Directorio \texttt{./Imagenes}] contiene las im�genes del
  documento. Dentro de �l aparecen varios directorios distintos. La
  gesti�n de im�genes (y por lo tanto la estructura de estos
  directorios) se describir� en el cap�tulo~\ref{cap:imagenes}.

\item[Directorio \texttt{./TeXiS}] contiene todos los ficheros
  relacionados con la propia plantilla, es decir, los ficheros que
  definen la apariencia final del documento, as� como los comandos que
  facilitan la edici�n que ser�n descritos en el
  cap�tulo~\ref{cap:edicion}. La creaci�n de un documento que se
  adhiere completamente al formato de \texis\ no necesitar� tocar
  ninguno de los ficheros de este directorio.

\item[Directorio \texttt{./VersionesPrevias}] Este directorio es
  usado por el \texttt{Makefile} cuando se realiza una copia de
  seguridad del estado del documento. Describiremos esta
  caracter�stica en el cap�tulo~\ref{cap:makefile}.
\end{description}

Existen por lo tanto, tres tipos de ficheros \texttt{.tex}: los
ficheros que contienen el texto principal del documento (cap�tulos y
ap�ndices), los ficheros que definen las partes adicionales del mismo
(como portada y agradecimientos), y los ficheros que determinan la
apariencia. En las tres secciones siguientes describimos cada uno de
ellos.

%-------------------------------------------------------------------
\section{Ficheros con el texto principal del documento}
%-------------------------------------------------------------------
\label{cap2:sec:ficheros-texto}

Estos \texttt{.tex} son los que contienen el texto tanto de los
cap�tulos como de los ap�ndices, por lo tanto son los ficheros que m�s
tiempo pasar�s editando. Est�n divididos en secciones, tienen figuras,
tablas, referencias bibliogr�ficas, y cualquier otro tipo de elemento
que quieras o debas a�adir.

En principio pueden contener cualquier c�digo \LaTeX. No obstante, no
olvides que si necesitas alg�n paquete especial que no se cargue por
defecto en la plantilla, deber�s incluir el \verb+\usepackage+
correspondiente en el documento maestro o en el fichero de pre�mbulo
de \texis, \url{TeXiS/TeXiS\_pream.tex} descrito en la
Secci�n~\ref{cap2:sec:ficheros-formato}.

El cap�tulo siguiente est� enteramente dedicado al proceso de edici�n
de estos ficheros.

%-------------------------------------------------------------------
\section[Ficheros del documento auxiliares]%
{Ficheros del documento auxiliares: las c�scaras del documento}
%-------------------------------------------------------------------
\label{cap2:sec:ficheros-auxiliares}

Estos ficheros, como ya hemos dicho, son los responsables del
contenido del resto de p�ginas del documento, todo aquello que no son
cap�tulos o ap�ndices. Son los siguientes (por orden de ``aparici�n''
en el documento final)\footnote{Si crees que no necesitas alguno de
  ellos, puedes eliminar su inclusi�n en el fichero maestro,
  \texttt{Tesis.tex}.}:

\begin{itemize}
\item \texttt{cover.tex}: responsable de las dos primeras hojas del
  documento, que forman las portada. Mediante comandos se definen el
  autor y t�tulo que aparecer� en la portada, la fecha de publicaci�n,
  facultad, etc. Como podr�s ver cuando lo edites, el fichero contiene
  los datos concretos para generar este manual. Los comandos se
  describen en la secci�n~\ref{cap3:sec:tituloYmetadatas}.

\item \texttt{dedicatoria.tex}: contiene el c�digo \LaTeX\ que crea la
  ``dedicatoria'' de la Tesis. Consiste en una hoja donde aparece
  alineada a la izquierda una frase indicando a qui�n se ``dedica'' el
  documento (en los libros serios pone algo como ``A mis padres'',
  aunque tambi�n hay autores en libros m�s distendidos, como
  \citeauthor{AIbyExample} que dice textualmente ``For Mum and Dad,
  who bought me my first computer, and therefore must share some
  responisibility for turning me into the geek that I am''
  \citep{AIbyExample}). Se pueden poner todas las p�ginas de
  dedicatorias que se deseen, utilizando la macro
  \verb|\putDedicatoria|, que recibe la cita completa y crea la hoja
  completa con la misma. Lo m�s c�modo, no obstante, es utilizar la
  macro \verb|\dedicatoriaUno| y (opcionalmente)
  \verb|\dedicatoriaDos| para establecer las dos dedicatorias y a
  continuaci�n invocar \verb|\makeDedicatorias| para generarlas. As�
  lo hace este manual.

\item \texttt{agradecimientos.tex}: contiene el texto de las �nicas
  p�ginas que tu familia y amigos van a leer de la Tesis: los
  agradecimientos. As� que piensa bien lo que pones, no olvides a
  nadie\footnote{Tampoco a nosotros por quitarte la preocupaci�n del
    aspecto final... \texttt{:-)}}.

  Es importante que no borres la l�nea que aparece justo despu�s del
  \verb|\chapter|,

\begin{verbatim}
\cabeceraEspecial{Agradecimientos}
\end{verbatim}

  ya que lo que hace es modificar la cabecera de la p�gina para que no
  aparezca con el mismo formato que en los cap�tulos. Puedes consultar
  la secci�n~\ref{cap3:ssec:capitulos-especiales} para obtener m�s
  detalles sobre esto.

\item \texttt{resumen.tex}: si quieres incluir antes del �ndice un
  peque�o resumen de tu trabajo, puedes hacerlo en este fichero. Al
  igual que en los agradecimientos no debes eliminar el comando
  \LaTeX\ del principio que altera la cabecera.

  Tanto el resumen como los agradecimientos antes explicados se
  convierten en dos ``cap�tulos sin numeraci�n'' que tambi�n ser�n
  listados en el �ndice de contenidos. No obstante, al aparecer antes
  que el texto principal del documento (los cap�tulos propiamente
  dichos), sus p�ginas ser�n numeradas con notaci�n romana, en
  lugar de con la ar�biga tradicional.

\item \texttt{bibliografia.tex}: en �l se configura la
  bibliograf�a del documento. En concreto, el fichero permite indicar
  tanto qu� ficheros \texttt{.bib} contienen las entradas
  bibliogr�ficas como una frase c�lebre (seguramente, ya habr�s notado
  que \texis\ permite iniciar los cap�tulos con una frase
  c�lebre), caracter�stica descrita con m�s detalle en la secci�n
  \ref{cap3:ssec:frases}.

  El cap�tulo~\ref{cap:bibliografia} hace una descripci�n m�s
  detallada del tipo de bibliograf�a que propone utilizar la plantilla
  (y que utiliza este manual).

\item \texttt{fin.tex}: En nuestras respectivas tesis, como ``cierre''
  incluimos una �ltima p�gina parecida a la dedicatoria con un par de
  frases c�lebres. El c�digo \TeX\ responsable se encuentra en este
  fichero.
\end{itemize}

Existen otros dos ficheros que no aparecen en este directorio pero que
generan p�ginas en el documento final. Son \texttt{TeXiS\_toc.tex} y
\texttt{TeXiS\_acron.tex} del directorio \texttt{TeXiS}, descritos en
la secci�n~\ref{cap2:sec:ficheros-formato}. Aparecen en ese directorio
debido a que no permiten ning�n tipo de personalizaci�n al usuario de
\texis.

%-------------------------------------------------------------------
\section{Directorio raiz}
%-------------------------------------------------------------------
\label{cap2:sec:directorio-raiz}

En el directorio ra�z aparecen, adem�s de \texttt{Tesis.tex}, el
documento maestro, otros tres ficheros \texttt{.tex} que no son
responsables de la generaci�n de ninguna p�gina del documento. Uno de
ellos, \texttt{config.tex} se describe en la
secci�n~\ref{cap3:sec:modos-compilacion}. Los otros dos son:

\begin{itemize}
\item \texttt{guionado.tex}: contiene una lista de aquellas palabras
  que, durante la edici�n del documento, se ha podido comprobar que
  \LaTeX\ divid�a mal. En esos casos, la alternativa mala es hacer
  peque�os ajustes en el p�rrafo para que esa palabra cuyos guiones
  \LaTeX\ no sabe colocar no quede cerca del final de la l�nea. La
  alternativa buena es a�adir la palabra a este fichero, colocando los
  guiones donde van. En el fichero proporcionado aparece una lista de
  algunas palabras de ejemplo.

\item \texttt{constantes.tex}: est� pensado para la definici�n de
  constantes que aparezcan a menudo en el texto. Por ejemplo, si se
  hace un documento sobre Cruise Control~\citep{CruiseControl}, para
  evitar tener que escribir cont�nuamente las dos palabras, es buena
  idea incluir una constante en el fichero que cree un comando para
  hacerlo m�s r�pidamente:

\begin{example}
\newcommand{\cc}{Cruise Control}
La nueva versi�n de \cc\ \ldots
\end{example}

En este fichero aparece definida la constante \verb|\titulo| que
contiene el t�tulo del documento y \verb|\autor| con el autor. Ambos
son utilizados en la portada. Tambi�n aparece definido el comando
\verb|\texis| que utilizamos en este manual para evitarnos escribir el
c�digo que escribe ``\texis''\ una y otra vez:

\begin{example}
\texis\ te permite generar el
fichero final tanto como .dvi
como en un .pdf.
\end{example}
\end{itemize}

Por �ltimo indicar que en el directorio ra�z aparecen los ficheros con
extensi�n \texttt{.bib} que contienen la informaci�n bibliogr�fica y
los \texttt{.gdf} para los acr�nimos (ver
cap�tulo~\ref{cap:bibliografia}) as� como el fichero \texttt{Makefile}
para la generaci�n autom�tica del documento final
(cap�tulo~\ref{cap:makefile}).

%-------------------------------------------------------------------
\section{Ficheros de la plantilla}
%-------------------------------------------------------------------
\label{cap2:sec:ficheros-formato}

El directorio \texttt{TeXiS} contiene los ficheros que definen la
apariencia final del documento. Si el formato de este manual te gusta
tal cual, no tendr�s por qu� tocar ninguno de estos ficheros. La
explicaci�n de su contenido aparece a continuaci�n. Su c�digo fuente
contiene numerosos comentarios y enlaces, por lo que no deber�a
suponerte demasiado problema modificarlos.

\begin{itemize}
\item \texttt{TeXiS\_cab.tex}: contiene la definici�n de la apariencia
  de las cabeceras de las p�ginas utilizadas en el documento. La
  plantilla utiliza el paquete \texttt{fancyhdr}. Sin embargo, la
  cabecera por defecto se ha modificado para que aparezca el n�mero
  del cap�tulo, as� como su nombre en min�sculas, junto con alg�n otro
  cambio menor.

\item \texttt{TeXiS.sty}: contiene los comandos que la plantilla
  proporciona para facilitar el proceso de edici�n. El uso de estos
  comandos est� explicado en el cap�tulo~\ref{cap:edicion}. A pesar de
  que la extensi�n distinta a la habitual (\texttt{.sty} en vez de
  \texttt{.tex}) puede imponer cierto respeto al principio, puedes
  abrir sin miedo el fichero para edici�n, pues es un fichero de
  \LaTeX\ normal, con definiciones de comandos tradicionales.

\item \texttt{TeXiS.bst}: contiene el estilo que utiliza la plantilla
  para generar la lista de las referencias bibliogr�ficas al final del
  documento. Las posibilidades de este estilo son descritas en el
  cap�tulo~\ref{cap:bibliografia}.

\item \texttt{TeXiS\_pream.tex}: este fichero contiene la mayor parte
  del c�digo del pre�mbulo del documento (lo que va antes del
  \verb|\begin{document}|). En �l aparecen incluidos un buen n�mero de
    paquetes que pueden ser �tiles en la elaboraci�n del documento,
    junto con una explicaci�n de para qu� sirven y, en algunas
    ocasiones, algunos ejemplos de uso. Existen incluso ciertos
    paquetes cuya inclusi�n aparece comentada pero que se mantienen,
    junto con su comentario correspondiente, por si pueden venir bien
    para documentos concretos que necesiten ciertas caracter�sticas
    que ni este manual ni nuestras tesis requirieron.

  \item \texttt{TeXiS\_cover.tex}: contiene el c�digo \TeX\ que genera
    la portada, y la hoja siguiente a la misma, que vuelve a tener los
    mismos datos pero sin el escudo.

  \item \texttt{TeXiS\_dedic.tex}: contiene el c�dito \TeX\ para
    generar las hojas de dedicatorias.

  \item \texttt{TeXiS\_toc.tex}: es el responsable de la generaci�n de
    los �ndices de cap�tulos, tablas y figuras que aparece en el
    documento.

  \item \texttt{TeXiS\_bib.tex}: es el encargado de que en el
    documento aparezca bibliograf�a. Incluido desde el fichero
    maestro, lo primero que hace es leer el fichero de configuraci�n,
    \texttt{Cascaras/configBibliografia.tex}.

    Como puedes comprobar, la bibliograf�a es tambi�n referenciada en
    el �ndice como un cap�tulo sin numerar; tambi�n se preocupa de
    cambiar la cabecera para que no se utilice la habitual del resto
    de cap�tulos.

\item \texttt{TeXiS\_acron.tex}: la plantilla tambi�n permite a�adir
  una lista de acr�nimos o abreviaturas utilizadas en el texto. En
  este fichero se incluyen los comandos necesarios para que aparezca
  esta lista. No obstante, para que la lista funcione, en el momento
  de la generaci�n se debe invocar a la herramienta correspondiente
  para que se creen los ficheros auxiliares necesarios para su
  generaci�n. En la descripci�n sobre la generaci�n dada en la
  secci�n~\ref{cap2:sec:compilacion} no se describe este proceso,
  por lo que el resultado contendr� una lista de acr�nimos
  vac�a. El uso de acr�nimos se describe con detalle en la
  secci�n~\ref{capBiblio:sec:glosstex}.

\end{itemize}
  
%-------------------------------------------------------------------
\section{Generando el documento}
%-------------------------------------------------------------------
\label{cap2:sec:compilacion}

Como ya se dijo en la introducci�n, \texis\ permite compilar el
documento\footnote{Cuando hablamos de ``compilaci�n'' nos referimos,
  por analog�a con el desarrollo software, a la generaci�n del fichero
  final (un PDF) resultado de analizar los ficheros fuente en \LaTeX.}
tanto con \verb+latex+ como \verb+pdflatex+.  Si has utilizado \LaTeX\
a trav�s de editores de texto espec�ficos (como Kile o WinEdt), es
posible que no sepas de qu� estamos hablando. Tanto \texttt{latex}
como \texttt{pdflatex} son dos aplicaciones que cogen un fichero
\texttt{.tex} como entrada y generan el documento final
``renderizado''. La diferencia entre ambas radica en el fichero de
salida que generan. En el primer caso, se genera un fichero
\texttt{.dvi}\footnote{\emph{Device independent}, o ``independiente
  del dispositivo'' (en el que se mostrar� el contenido).}, mientras
que en el segundo caso se genera un fichero PDF directamente.
Tradicionalmente se ha utilizado \texttt{latex}, convirtiendo despu�s
el fichero \texttt{.dvi} al formato deseado (como \texttt{.ps} o
\texttt{.pdf}). Sin embargo, en nuestro caso, aconsejamos la
utilizaci�n de \texttt{pdflatex}, debido a que, al generar de forma
nativa ficheros PDF, aprovecha algunas de las caracter�sticas
disponibles en los mismos. En particular, \texis\ contiene algunos
comandos \LaTeX\ que \texttt{pdflatex} aprovecha para a�adir
informaci�n de \emph{copyright} al fichero, as� como enlaces a cada
uno de los cap�tulos y secciones del documento, permitiendo una
navegaci�n r�pida por el mismo cuando se utilizan visores
(figura~\ref{cap2:fig:pdf}).

\begin{figure}[t]
  \centering
  %
  \subfloat[][Propiedades del documento]{
     \includegraphics[width=0.42\textwidth]%
                     {Imagenes/Bitmap/02/PropiedadesPDF}
     \label{cap2:fig:PropiedadesPDF}
  }
  \qquad
  \subfloat[][Tabla de contenidos]{
     \includegraphics[width=0.42\textwidth]%
                     {Imagenes/Bitmap/02/IndicePDF}
     \label{cap2:fig:TocPDF}
  }
 \caption{Capturas del visor de PDF\label{cap2:fig:pdf}}
\end{figure}


La plantilla incluye un fichero \texttt{Makefile} para automatizar la
generaci�n del fichero final\footnote{Los ficheros \texttt{Makefile}
  son ampliamente utilizados en el desarrollo de software. Son
  ficheros que sirven de entrada a la utilidad \texttt{make} que
  genera autom�ticamente los ficheros de resultado a partir de los
  archivos de c�digo fuente.} que es capaz de crear el PDF utilizando
cualquiera de las dos alternativas. No obstante, en este apartado no
entraremos en los detalles de este fichero, ya que existe un cap�tulo
dedicado enteramente a �l (cap�tulo~\ref{cap:makefile}).

Para generar el documento de este manual a partir de los ficheros de
\texis\ proporcionados, la forma inmediata es seguir el proceso
tradicional de generaci�n de cualquier fichero de \LaTeX, es decir,
ejecutar \texttt{pdflatex} (o \texttt{latex}), a continuaci�n ejecutar
\texttt{bibtex} para resolver las referencias bibliogr�ficas, y
posteriormente ejecutar un par de veces m�s \texttt{pdflatex} para
resolver las referencias cruzadas y que aparezcan en el documento
final.

En l�nea de comandos eso se traduce a las siguientes
�rdenes\footnote{Tambi�n es v�lido el uso de \texttt{latex} en lugar
  de \texttt{pdflatex}, pero el fichero generado (\texttt{.dvi})
  deber� despu�s ser convertido a PDF.}:

\begin{verbatim}
$ pdflatex Tesis
$ bibtex Tesis
$ pdflatex Tesis
$ pdflatex Tesis
\end{verbatim}

Si se utiliza alg�n editor de \LaTeX\ para la edici�n, tambi�n se
pueden utilizar sus teclas r�pidas (o en su defecto, sus botones u
opciones de men�) para generarlo; encontrar�s una explicaci�n al
respecto en la secci�n~\ref{cap3:sec:editores}.

%-------------------------------------------------------------------
\section*{\NotasBibliograficas}
%-------------------------------------------------------------------
\TocNotasBibliograficas

En este cap�tulo hemos descrito simplemente la estructura de
directorio de \texis, por lo que no existe ninguna fuente
relacionada adicional de consulta. Se mantiene este apartado por
simetr�a con el resto de cap�tulos. En un documento normal (tesis,
trabajo de investigaci�n) lo m�s probable es que todos los cap�tulos
puedan extenderse con notas de este tipo.

%-------------------------------------------------------------------
\section*{\ProximoCapitulo}
%-------------------------------------------------------------------
\TocProximoCapitulo

Una vez que se han descrito a vista de p�jaro los ficheros que
componen la plantilla y una primera aproximaci�n al proceso de
generaci�n del documento final (en PDF), el siguiente cap�tulo pasa a
describir el proceso de edici�n.

Eso cubre aspectos tales como los ficheros que deben modificarse para
a�adir nuevos cap�tulos o los comandos que \texis\ hace
disponibles para escribir ciertas partes de los mismos. El cap�tulo
describe tambi�n los dos modos de generaci�n del documento final que
pueden ser de utilidad durante el largo proceso de escritura. Por
�ltimo, el cap�tulo terminar� con ciertas consideraciones relativas a
los editores de \LaTeX\ utilizados as� como sobre la posibilidad de
utilizar un control de versiones.

% Variable local para emacs, para  que encuentre el fichero maestro de
% compilaci�n y funcionen mejor algunas teclas r�pidas de AucTeX
%%%
%%% Local Variables:
%%% mode: latex
%%% TeX-master: "../ManualTeXiS.tex"
%%% End:

...

% Apéndices
\appendix
%---------------------------------------------------------------------
%
%                          Ap�ndice 1
%
%---------------------------------------------------------------------
%
% 01AsiSeHizo.tex
% Copyright 2009 Marco Antonio Gomez-Martin, Pedro Pablo Gomez-Martin
%
% This file belongs to the TeXiS manual, a LaTeX template for writting
% Thesis and other documents. The complete last TeXiS package can
% be obtained from http://gaia.fdi.ucm.es/projects/texis/
%
% Although the TeXiS template itself is distributed under the 
% conditions of the LaTeX Project Public License
% (http://www.latex-project.org/lppl.txt), the manual content
% uses the CC-BY-SA license that stays that you are free:
%
%    - to share & to copy, distribute and transmit the work
%    - to remix and to adapt the work
%
% under the following conditions:
%
%    - Attribution: you must attribute the work in the manner
%      specified by the author or licensor (but not in any way that
%      suggests that they endorse you or your use of the work).
%    - Share Alike: if you alter, transform, or build upon this
%      work, you may distribute the resulting work only under the
%      same, similar or a compatible license.
%
% The complete license is available in
% http://creativecommons.org/licenses/by-sa/3.0/legalcode
%
%---------------------------------------------------------------------

\chapter{As� se hizo...}
\label{ap1:AsiSeHizo}

\begin{FraseCelebre}
\begin{Frase}
Pones tu pie en el camino y si no cuidas tus pasos, nunca sabes a donde te pueden llevar.
\end{Frase}
\begin{Fuente}
John Ronald Reuel Tolkien, El Se�or de los Anillos
\end{Fuente}
\end{FraseCelebre}

\begin{resumen}
Este ap�ndice cuenta algunos aspectos pr�cticos que nos planteamos en
su momento durante la redacci�n de la tesis (a modo de ``as� se hizo
nuestra tesis''). En realidad no es m�s que una excusa para que �ste
manual tenga un ap�ndice que sirva de ejemplo en la plantilla.
\end{resumen}

%-------------------------------------------------------------------
\section{Edici�n}
%-------------------------------------------------------------------
\label{ap1:edicion}

Ya indicamos en la secci�n~\ref{cap3:sec:editores} (p�gina
\pageref{cap3:sec:editores}) que \texis\ est� preparada para
integrarse bien con emacs, en particular con el modo Auc\TeX.

Eso era en realidad un s�ntoma indicativo de que en nuestro trabajo
cotidiano utilizamos emacs para editar ficheros \LaTeX. Es cierto que
inicialmente utilizamos otros editores creados expresamente para la
edici�n de ficheros en \LaTeX, pero descubrimos emacs y ha llegado
para quedarse (la figura~\ref{cap3:fig:emacs} mostraba una captura del
mismo mientras cre�bamos este manual). Ten en cuenta que si utilizas
Windows, tambi�n puedes usar emacs para editar; no lo consideres como
algo que s�lo se utiliza en el mundo Unix. Nosotros lo usamos a diario
tanto en Linux como en Windows.

No obstante, hay que reconocer que emacs \emph{no} es f�cil de
utilizar al principio (el manual de referencia de \cite{emacsStallman}
tiene m�s de 550 p�ginas); su curva de aprendizaje es empinada,
especialmente si quieres sacarle el m�ximo partido, o al menos
beneficiarte de algunas de sus combinaciones de teclas. Pero una vez
que consigues \emph{no} mover las manos para desplazar el cursor sobre
el documento, manejas las teclas r�pidas para a�adir los comandos
\LaTeX\ m�s utilizados y conoces las combinaciones de Auc\TeX\ para
moverte por el documento o buscar las entradas de la bibliograf�a, no
cambiar�s f�cilmente a otro editor.

Si quieres aprovechar emacs, no debes dejar de leer el documento que
nos introdujo a nosotros en el modo Auc\TeX, ``\emph{Creaci�n de
  ficheros \LaTeX\ con GNU Emacs}'' \citep{AtazLopezEmacs}.

%-------------------------------------------------------------------
\section{Encuadernaci�n}
%-------------------------------------------------------------------
\label{ap1:encuadernacion}

Si has mirado con un poco de atenci�n este manual, habr�s visto que
los m�rgenes que tiene son bastante grandes. \texis\ no configura
los m�rgenes a unos valores concretos sino que, directamente, utiliza
los que se establecen por defecto en la clase \texttt{book} de \LaTeX.

Aunque es m�s o menos reconocido que si \LaTeX\ utiliza esos m�rgenes
debe tener una raz�n de peso (y de hecho la tiene, se utilizan esos
para que el n�mero de letras por l�nea sea el id�neo para su lectura),
cuando se comienza a mirar el documento con los ojos del que quiere
verlo encuadernado, es cierto que parecen excesivos. Y empiezas a
abrir libros, regla en mano, para medir qu� m�rgenes utilizan. Y
reconoces que son mucho m�s peque�os (y razonables) que el de tu
maravilloso escrito. Al menos ese fue nuestro caso.

En ese momento, una soluci�n es \emph{reducir} esos m�rgenes para que
aquello quede mejor. Sin embargo nuestra opci�n no fue esa. Si tu
situaci�n te permite \emph{no} encuadernar el documento en formato
DIN-A4, entonces puedes ir a la reprograf�a de turno y pedir que, una
vez impreso, te guillotinen esos m�rgenes.

Tu escrito quedar� entonces en ``formato libro'', mucho m�s manejable
que el gran DIN-A4, y con unos m�rgenes mucho m�s razonables. La
figura~\ref{ap1:fig:encuadernacion} muestra el resultado, comparando
el tama�o final con el de un folio, que aparece superpuesto.

\figura{Bitmap/0A/encuadernacion}{width=0.7\textwidth}%
       {ap1:fig:encuadernacion}{Encuadernaci�n y m�rgenes guillotinados}

%-------------------------------------------------------------------
\section{En el d�a a d�a}
%-------------------------------------------------------------------
\label{ap1:cc}

Para terminar este breve ap�ndice, describimos ahora un modo de
trabajo que, si bien no utilizamos en su d�a para la escritura de la
tesis, s� hemos utilizado desde hace alg�n tiempo para el resto de
nuestros escritos de \LaTeX , incluidos \texis\ y �ste, su manual.

Estamos hablando de lo que se conoce en el mundo de la ingenier�a del
software como \emph{integraci�n cont�nua} \citep{Fowler06}. En
concreto, la integraci�n cont�nua consiste en aprovecharse del
servidor del control de versiones para realizar, en cada
\emph{commit} o actualizaci�n realizada por los autores, una
comprobaci�n de si los ficheros que se han subido son de verdad
correctos.

En el mundo del desarrollo software donde un proyecto puede involucrar
decenas de personas realizando varias actualizaciones diarias, la
integraci�n cont�nua tiene mucha importancia. Despu�s de que un
programador realice una actualizaci�n, un servidor dedicado comprueba
que el proyecto sigue compilando correctamente (e incluso ejecuta los
test de unidad asociados). En caso de que la actualizaci�n haya
estropeado algo, el servidor de integraci�n env�a un mensaje de correo
electr�nico al autor de ese \emph{commit} para avisarle del error y
que �ste lo subsane lo antes posible, de forma que se perjudique lo
menos posible al resto de desarrolladores.

Esa misma idea la hemos utilizado en la elaboraci�n de \texis\ y
de este manual. Cada vez que uno de los autores sub�a al SVN alg�n
cambio, el servidor comprobaba que el fichero maestro segu�a siendo
correcto, es decir, que se pod�a generar el PDF final sin errores.

No entraremos en m�s detalles de c�mo hacer esto. El lector interesado
puede consultar \citet{CCLatex}. Como se explica en ese art�culo
algunas ventajas del uso de esta t�cnica son:

\begin{figure}[t]
  \centering
  %
  \subfloat[][P�gina de descarga del documento generado]{
     \includegraphics[width=0.445\textwidth]%
                     {Imagenes/Bitmap/0A/dashboard}
     \label{ap1:fig:dashboard}
  }
  \qquad
  \subfloat[][M�tricas del proyecto]{
     \includegraphics[width=0.445\textwidth]%
                     {Imagenes/Bitmap/0A/metrics}
     \label{ap1:fig:metrics}
  }
 \caption{Servidor de integraci�n cont�nua\label{ap1:fig:cc}}
\end{figure}

\begin{itemize}
\item Se tiene la seguridad de que la versi�n disponible en el control
  de versiones es v�lida, es decir, es capaz de generar sin errores el
  documento final.

\item Se puede configurar el servidor de integraci�n cont�nua para que
  cada vez que se realiza un \emph{commit}, env�e un mensaje de correo
  electr�nico \emph{a todos los autores} del mismo. De esta forma
  todos los colaboradores est�n al tanto del progreso del mismo.

\item Se puede configurar para que el servidor haga p�blico (via
  servidor Web) el PDF del documento (ver
  figura~\ref{ap1:fig:dashboard}). Esto es especialmente �til para
  revisores del texto como tutores de tesis, que no tendr�n que
  preocuparse de descargar y compilar los \texttt{.tex}.
\end{itemize}

Por �ltimo, el servidor tambi�n permite ver la evoluci�n del proyecto.
La figura~\ref{ap1:fig:metrics} muestra una gr�fica que el servidor de
integraci�n cont�nua muestra donde se puede ver la fecha (eje
horizontal) y hora (eje vertical) de cada \emph{commit} en el
servidor; los puntos rojos representan commits cuya compilaci�n fall�.



% Variable local para emacs, para  que encuentre el fichero maestro de
% compilaci�n y funcionen mejor algunas teclas r�pidas de AucTeX
%%%
%%% Local Variables:
%%% mode: latex
%%% TeX-master: "../ManualTeXiS.tex"
%%% End:

...
\end{verbatim}

\medskip

Todos estos ficheros de capítulos y apéndices deben comenzar con el
comando \LaTeX\ \verb|\chapter|\footnote{Esto \emph{también} se
  cumple para los apéndices.}. El resto del fichero es un fichero
\LaTeX\ normal que tendrá secciones, subsecciones, figuras, tablas,
etc.

Al añadir un nuevo fichero, es posible que también quieras añadir su
nombre en el fichero \texttt{config.tex} para permitir la compilación
rápida de un único capítulo según se cuenta en la
seccion~\ref{cap3:sec:compilacion-rapida}.

%-------------------------------------------------------------------
\subsection{Resumen del capítulo}
%-------------------------------------------------------------------

\texis\ permite incluir al comienzo de todos los capítulos un
breve resumen del mismo; este mismo manual lo hace. Para separarlo del
resto se utiliza un formato distinto.

En vez de cambiar el formato en todos y cada uno de los capítulos (y
apéndices), \texis\ proporciona un \emph{entorno} nuevo,
\texttt{resumen}, que lo hace por nosotros:

\begin{example}
\begin{resumen}
En este capítulo se describe...
\end{resumen}
\end{example}

El formato concreto está definido en el fichero
\texttt{TeXiS/TeXiS.sty}, por lo que se puede cambiar a voluntad, lo
que provocará el cambio en todas sus apariciones.

%-------------------------------------------------------------------
\subsection{Frases célebres}
%-------------------------------------------------------------------
\label{cap3:ssec:frases}

Como habrás podido comprobar leyendo este manual, \texis\ permite
además escribir en cada capítulo una ``frase célebre'' que es añadida
inmediatamente después del título del mismo, alineada a la derecha.

Para añadir la frase (que está formada por la cita en cuestión y su
autor), \texis\ define un nuevo entorno \texttt{FraseCelebre},
dentro del cual se especifican cada una de ellas con otros dos
entornos, \texttt{Frase} y \texttt{Fuente}:

\begin{example}
\begin{FraseCelebre}
\begin{Frase}
Nadie espere que yo diga algo.
\end{Frase}
\begin{Fuente}
Mafalda
\end{Fuente}
\end{FraseCelebre}
\end{example}

Evidentemente, las frases célebres pueden añadirse en todos los
capítulos, incluidos los ``especiales'' (aquellos que no tienen
numeración normal) como el capítulo de agradecimientos. Para hacerlo,
basta con utilizar los comandos anteriores.

Un capítulo donde es algo más complicado es el ``capítulo'' de
\emph{bibliografía}. Esto es debido a que la generación del capítulo
completo consiste en una mera invocación al comando
\verb+bibliography+

\begin{verbatim}
\bibliography{fichero1,fichero2}
\end{verbatim}

En el \verb|documentclass| que estamos utilizando (\texttt{book}) eso
significa que se creará un nuevo \emph{capítulo} con la lista de
referencias. Si en ese capítulo se quiere añadir una cita (como
hacemos por ejemplo en este manual), hay que realizar algunas tareas
adicionales. Naturalmente \texis\ las hace por nosotros, por lo que,
como se mencionó en la sección~\ref{cap2:sec:ficheros-auxiliares}, lo
único que tendremos que hacer es editar el fichero
\texttt{bibliografia.tex}, buscar la frase célebre del manual y
cambiarla a voluntad.

Antes de terminar, decir que, igual que en el caso del resumen, la
apariencia de la frase célebre se puede modificar en el fichero
\texttt{TeXiS/TeXiS.sty}.

%-------------------------------------------------------------------
\subsection{Secciones no numeradas}
%-------------------------------------------------------------------
\label{cap3:ssec:secciones-no-numeradas}

Como habrás podido comprobar, en este manual todos los capítulos
terminan con dos secciones no numeradas, una de ellas con unas notas
bibliográficas, y otra que tiene un pequeño resumen del siguiente
capítulo.

Aunque para el manual no son en realidad necesarias (especialmente la
de notas bibliográficas, que en muchos capítulos nos ha costado
rellenar\ldots), las hemos puesto para que sirvan de ejemplo en el
\texttt{.tex}.

En principio, para poner una sección no numerada basta con utilizar la
``versión estrellada'' del comando \LaTeX\ correspondiente. Es decir,
utilizar \verb+\section*+ para añadir una sección sin número. El
problema en nuestro caso es que este comando no parece funcionar
correctamente con el paquete \texttt{fancyhdr}. \texis\ utiliza
ese paquete para configurar la cabecera y pie de página; en concreto
para indicar que se desea que el número de página aparezca en las
esquinas ``externas'', mientras que en las esquinas internas debe
aparecer el nombre del capítulo (en las hojas pares o izquierdas) y
sección (en las impares o derechas). El mismo paquete es el que se
utiliza para que aparezca el número de página en la primera página de
un capítulo y para cierta información que aparece cuando se genera el
documento en ``modo borrador'', según aparece descrito en la
sección~\ref{cap3:sec:modos-compilacion}.

El problema aparece cuando una sección no numerada excede
el límite de la página en la que empieza. En ese caso, la cabecera en
la que aparece el nombre de la sección en vez de contener el título de
esa sección sin numerar, seguirá mostrando la última sección numerada.

La solución es modificar a mano la cabecera, en concreto modificar la
configuración de la cabecera donde aparece el título de la sección
actual (la parte izquierda de las páginas impares). Para eso, tras
consultar la documentación del paquete, se aprende que hay que
utilizar el comando \verb+\markright+. Por ejemplo:

\begin{verbatim}
\section*{Notas bibliográficas\markright{Notas bibliográficas}}
\end{verbatim}

Como puede verse, en el propio comando \verb+\section*+, se incluye una
llamada a \verb+\markright+, que contiene el texto que debe aparecer a
en la cabecera. Con esto se soluciona el problema de las cabeceras.

Otro ``problema'' de las secciones sin numerar es que no se meten en
la tabla de contenidos que se incluye al principio del documento;
tampoco aparecen en el ``contenido'' del PDF listado por el visor que
mostrabamos en la figura~\ref{cap2:fig:pdf}\footnote{Ponemos
  \emph{problema} entre comillas porque normalmente se utiliza la
  versión con estrella de los comandos \texttt{section} precisamente
  para evitar que una sección aparezca en el índice.}. Sin embargo, en
nuestro caso preferíamos que también las secciones aparecieran en el
índice (es decir, que la única diferencia entre las secciones
numeradas y las no numeradas fuera, precisamente, la ausencia de
numeración). Para que aparezca, por lo tanto, se debe añadir
explícitamente la sección en la tabla de contenidos, con el comando:

\begin{verbatim}
\addcontentsline{toc}{section}{Notas bibliográficas}
\end{verbatim}

\noindent que debe ejecutarse \emph{después} del comando
\verb+\section*+. Por lo tanto, para añadir una sección sin numerar
como la de ``Notas bibliográficas'', el código \LaTeX\ final que hay
que poner es:

\begin{verbatim}
%--------------------------------------------------------------
\section*{Notas bibliográficas\markright{Notas bibliográficas}}
%--------------------------------------------------------------
\addcontentsline{toc}{section}{Notas bibliográficas}
\end{verbatim}

Entendemos que invocar a los comandos anteriores cada vez que se desea
una de estas secciones no numeradas es tedioso. Por ello \texis\
proporciona una serie de comandos (definidos en el fichero
\texttt{./TeXiS/TeXiS\_cab.tex}) que permiten añadir fácilmente cuatro
tipos de secciones sin numerar. Las secciones son los siguientes (ver
tabla~\ref{cap3:tab:seccionesnonumeradas}):

\begin{table}[t]
\footnotesize
\centering
\begin{tabular}{|l|c|c|}
\hline
Texto & Comando para \texttt{section} & Comando para índice \\
\hline
\hline
Conclusiones & \verb+\Conclusiones+ & \verb+\TocConclusiones+ \\
\hline
En el próximo capítulo\ldots & \verb+\ProximoCapitulo+ & \verb+\TocProximoCapitulo+ \\
\hline
Notas bibliográficas & \verb+\NotasBibliograficas+ & \verb+\TocNotasBibliograficas+ \\
\hline
Resumen & \verb+\Resumen+ & \verb+\TocResumen+ \\
\hline
\end{tabular}
\caption{Secciones no numeradas soportadas por \texis
   \label{cap3:tab:seccionesnonumeradas}}
\end{table}

\begin{itemize}
\item ``Conclusiones'': el manual no utiliza esta sección sin numerar,
  pero sí puede ser razonable utilizarlo a modo de resumen al final
  del capítulo de otro tipo de documentos. 

\item ``Notas bibliográficas'': también utilizado en este documento,
  es útil para dar otras referencias bibliográficas que por cualquier
  razón no se citó en el texto.

\item ``En el próximo capítulo...'': sí se ha utilizado en el manual,
  y puede servir para enlazar el contenido del capítulo con el
  siguiente.

\item ``Resumen'': con un objetivo parecido al de conclusiones pero
  con distinto título; tampoco lo utilizamos en el manual.
\end{itemize}

Como se puede ver en la tabla, para cada una de estas secciones
aparecen dos comandos, uno para el comando \verb+\section*+ y otro
para añadir el índice, de forma que la definición de, por ejemplo, la
sección de ``En el próximo capítulo...'' quedaría:

\begin{verbatim}
%--------------------------------------------------------------
\section*{\ProximoCapitulo}
%--------------------------------------------------------------
\TocProximoCapitulo
\end{verbatim}

Somos conscientes de que los dos comandos podrían haberse unificado en
uno sólo, como \verb+\SeccionProximoCapitulo+ y que él mismo hiciera todo el
trabajo (es decir, pusiera el \verb+\section*{...}+ así como el
\verb+\addcontestline+). Sin embargo, esta solución no es compatible
con la capacidad de los editores de resaltar secciones, ya que los
editores simplemente buscan la cadena ``\verb+\section+'' para
resaltarlo (ver figura~\ref{cap3:fig:emacs}).

\figura{Bitmap/03/SeccionesEmacs}{width=0.7\textwidth}%
       {cap3:fig:emacs}{Resaltado de secciones en emacs}

Es por ello que, a pesar de ser más tedioso, optamos por la
alternativa complicada: si se quiere meter una sección sin numerar, se
debe primero utilizar el comando \verb+\section*+, añadiendo como
texto el comando que aparece en la segunda columna de la
tabla~\ref{cap3:tab:seccionesnonumeradas}, y posteriormente se utiliza
el otro comando para añadirlo al índice. Separándolo así, además,
permite al usuario de \texis\ decidir si quiere o no que la
sección aparezca en el índice.

%-------------------------------------------------------------------
\subsection{Capítulos especiales}
%-------------------------------------------------------------------
\label{cap3:ssec:capitulos-especiales}

Relacionado con las cabeceras de la sección anterior, \texis\
soporta (y este manual tiene) capítulos ``especiales'' que aparecen
sin numerar. Estos ``capítulos'' son, en concreto, la parte de
agradecimientos y resumen, los índices y la bibliografía.

Dado que todos ellos se caracterizan por la ausencia de secciones, no
tiene sentido mantener la cabecera utilizada en el resto del texto.
Por lo tanto, configuramos sus cabeceras para que en ambas páginas
aparezca directamente el título del capítulo (también sin número).

Para hacerlo, \texis\ dispone del comando
\verb+\cabeceraEspecial+, que recibe como parámetro el nombre del
capítulo. De esta forma, el capítulo de agradecimientos comienza con:

\begin{verbatim}
\chapter{Agradecimientos}

\cabeceraEspecial{Agradecimientos}

\begin{FraseCelebre}
...
\end{verbatim}

que provoca un cambio en la cabecera que se debe utilizar.

Los capítulos sin numerar de este manual se encargan de configurar la
propia cabecera por lo que si partes de ellos para escribir tu
documento no deberás preocuparte de nada (más allá de \emph{no} borrar
el comando).

Si incluyes nuevos capítulos sin numerar, has de saber que:

\begin{itemize}
\item No debes olvidar invocar el comando anterior al principio del
  capítulo sin numerar.

\item El comando anterior \emph{sobreescribe} el funcionamiento normal
  de la cabecera, por lo que se debe llamar al comando
  \verb+\restauraCabecera+ para reestablecerlo \emph{después} del
  capítulo especial. Es importante resaltar el \emph{después} pues
  debe hacerse cuando el capítulo \emph{ya ha terminado} y o bien se
  ha empezado el siguiente o bien se ha forzado el final de página con
  un \verb+\newpage+. \texis\ ya hace esto automáticamente justo
  antes del primer capítulo (en \texttt{Tesis.tex}). Sin embargo, si
  incluyes algún capítulo especial más adelante en el documento, no
  debes olvidar restaurar la cabecera.
\end{itemize}

%-------------------------------------------------------------------
\subsection{Dividiendo el documento en partes}
%-------------------------------------------------------------------
\label{cap3:ssec:partes}

En ocasiones la estructura del documento tiene dos o más partes
claramente diferenciadas. Por ejemplo un libro puede tener una primera
parte de conceptos básicos con unos pocos capítulos y otra de
conceptos avanzados con el resto.

\LaTeX\ permite especificar distintas partes utilizando el comando
\verb|\part|. El resultado es la inserción de una
nueva hoja con el número (en romanos) y título de la parte y la
adaptación del índice de contenidos para incluir la información de esa
nueva parte.

Obviamente, \texis\ también permite la inclusión de distintas partes
(y este manual las tiene a modo de ejemplo). Sin embargo, en vez de
utilizar directamente el comando de \LaTeX, aconsejamos el uso de
comandos del propio \texis\ que tienen funcionalidad adicional.

En concreto, los comandos de \texis\ relacionados con las partes del
documento (y que describiremos a continuación) permiten añadir una
pequeña descripción de la parte que comienza en su hoja de título y
una  descripción más larga en la parte trasera (sólo si el documento
está configurado ``a dos caras'', especificando \texttt{twoside} en el
\texttt{documentclass} del  principio del documento).

\texis\ también se preocupa de que en el índice de contenidos del PDF
final la bibliografía (y en caso de existir la última hoja con la
frase célebre) \emph{no} aparezcan ligados a la última parte del
documento, sino que estén en su mismo nivel.

Dicho todo esto, aconsejamos que, igual que se hace en el código de
este manual, existan ficheros para definir cada una de las partes (en
el manual se llaman \texttt{Capitulos/Parte1.tex}, etc.). Estos
ficheros se incluyen desde el documento maestro justo antes del primer
capítulo de esa parte.

Los comandos de \texis\ relacionados con las partes del documento son
cuatro:

\begin{itemize}
\item \verb|\partTitle|: permite especificar el título de la parte que
  comenzará.

\item \verb|\partDesc|: para indicar el texto descriptivo que
  aparecerá en la ``portada'' de esa parte. Es opcional; si no se
  indica, no aparecerá descripción.

\item \verb|\partBackText|: sirve para especificar el texto que
  aparecerá en la parte trasera de la hoja que delimita esa nueva
  parte. Es responsbilidad del autor asegurarse de que ese texto entra
  perfectamente en una única cara. Igual que el anterior, es opcional.

\item \verb|\makepart|: tras indicar el título y, opcionalmente,
  descripción y texto trasero, este comando construye la hoja que
  define esa parte del documento. Si se desea crear una parte sin
  numerar (lo que en \LaTeX\ suele conseguirse con la versión ``con
  estrella'' del comando), se puede utilizar \verb|\makespart| (la
  \texttt{s} solicita la versión \emph{starred}).
\end{itemize}

A modo de ejemplo este manual contiene tres partes; la primera de ella
cubre los tres primeros capítulo y tiene tanto descripción como texto
en la parte trasera. La segunda tiene únicamente una descripción y la
tercera y última, para los apéndices, no tiene ni descripción ni texto
trasero.

El código \LaTeX\ para la definición de la primera parte es:

\begin{verbatim}
\partTitle{Conceptos básicos}

\partDesc{Esta primera parte del manual presenta los conceptos 
  básicos de \texis. Contiene un capítulo de introducción, 
  seguido de una descripción de la estructura de \texis\ y
  cómo se genera el documento final, para terminar con un
  capítulo en el que se describe el proceso de edición sugerido
  y los comandos que \texis\ proporciona para facilitar dicho
  proceso.}

\partBackText{En realidad la división por partes del manual no
  aporta demasiado al lector; se ha dividido en varias partes
  debido a que, en la práctica, el código de este manual sirve
  como ejemplo de uso de \texis.

  En un contexto distinto, es posible que un manual de este
  tipo no habría tenido estas partes así de diferenciadas.}

\makepart
\end{verbatim}


%-------------------------------------------------------------------
\section{Programando en \LaTeX}
%-------------------------------------------------------------------
\label{cap3:sec:programando}

Uno de los aspectos que diferencia a \LaTeX\ de los sistemas
ofimáticos tradicionales para creación de documentos es el modelo
subyacente que utiliza. En realidad, todo lo que el autor escribe en
sus ficheros \LaTeX\ es ``\emph{ejecutado}'' por el intérprete de
\LaTeX\ hasta generar el documento final. Por lo tanto, se puede decir
que básicamente, cuando se escribe en \LaTeX\ se ``está programando''
lo que posteriormente será un programa que generará nuestro documento
final. Afortunadamente esa sensación de ``programador'' no se tiene en
condiciones normales durante el proceso de autoría. Sin embargo esta
peculiaridad sí se puede aprovechar para facilitar el proceso de
edición.

Ya hemos visto en el capítulo anterior un ejemplo de cómo la
posibilidad de crear \emph{comandos} de \LaTeX\ nos permite establecer
``constantes'' que nos evitan tener que escribir palabras que
utilizaremos a menudo durante el texto. Sin embargo, profundizando un
poco más en el ``lenguaje'' que hay por debajo (por debajo de \LaTeX\
está \TeX) se puede comprobar que pone a nuestra disposición algunas
estructuras conocidas por los programadores como los \texttt{if}.

%-------------------------------------------------------------------
\section{Modos de generación del documento}
%-------------------------------------------------------------------
\label{cap3:sec:modos-compilacion}

Aprovechando esto, \texis\ está preparada para admitir dos
\emph{configuraciones de generación} o ``\emph{compilación}''
distintas que, imitando los nombres tradicionales en el desarrollo
software, llamamos configuración en modo ``release'' y en modo
``debug'' (o de depuración):

\begin{itemize}
\item La configuración en modo ``release'' está pensada para la
  versión ``definitiva'', por lo que genera un fichero con la
  apariencia final del documento.

\item La configuración en modo ``debug'' puede verse como una versión
  ``borrador''. En este caso el documento incluye ciertos elementos
  que no se desea incluir en la versión final, como comentarios en el
  propio texto.
\end{itemize}

La existencia de estos dos modos de compilación puede sonar extraña al
principio. En realidad, su utilidad depende del modo de escribir el
documento de cada uno. En nuestro caso, los capítulos de la tesis se
escribieron en un proceso ``iterativo'' de tal forma que incluíamos
comentarios que queríamos que aparecieran al imprimir ``la versión de
depuración'', pero no queríamos preocuparnos de tener que recordar
borrar llegado el momento de imprimir la versión final. Por otro lado,
cuando el documento es escrito por más de un autor (como este manual),
la posibilidad de poner comentarios fácilmente descartables es
especialmente útil.

Los ficheros descargados están configurados para compilar la versión
definitiva; para cambiarla a la versión de ``depuración'', basta con
cambiar el fichero \texttt{config.tex} del directorio raíz. En cierto
momento al principio del fichero aparecen las líneas siguientes.

\begin{verbatim}
% Comentar la línea si no se compila en modo release.
% TeXiS hará el resto
\def\release{1}
\end{verbatim}

Para generar el fichero con la configuración de depuración, basta con
comentar la línea en la que se ``define'' el símbolo
\texttt{release}\footnote{El comando recuerda a la orden del
  preprocesador de C/C++ ``\texttt{\#define release 1}''.}.

El primer efecto inmediato es que la plantilla añade automáticamente
como pie de página el texto:

\medskip

{\small \sc Borrador -- \today}

\medskip

De esta forma, si tienes varias versiones imprimidas puedes estar
tranquilo de que no se te mezclarán, pues además de marcar que es un
borrador, aparece la fecha en la que se generó el fichero.

En los tres apartados siguientes se describen tres comandos definidos
por \texis\ cuyo comportamiento depende del modo de compilación.

\subsection{Comando \texttt{com}}
\label{cap3:subsec:comando-com}

El comando \verb|\com| permite añadir un comentario que aparecerá (en
modo depuración) en un párrafo aparte, con un ancho de línea algo
superior a lo normal y rodeado de un cuadro negro.

% En este caso no podemos utilizar el entorno example, porque para
% poner el comentario de verdad no podemos utilizar el comando real,
% \com, pues en la compilación en release no saldría. He intenado hace
% que quede con la misma apariencia que en el ejemplo, pero no lo he
% logrado :(

Como ejemplo, el código \LaTeX:

\begin{verbatim}
\com{Lo que sigue podría en realidad ser una sección distinta...}
\end{verbatim}

Se convierte en:

\comImpl{Lo que sigue podría en realidad ser una sección distinta...}

Hay que advertir que el recuadro anterior no tiene ningún control
sobre los saltos de página, por lo que ante comentarios demasiado
grandes (que no entran en lo que queda de página), provoca que se
salte el resto de la misma y aparezca el comentario en la siguiente.

\subsection{Comando \texttt{comp}}
\label{cap3:subsec:comando-comp}

El comando anterior es muy útil pero debido a su tamaño puede no ser
recomendable para pequeños comentarios ``integrados'' dentro de un
párrafo. Para eso existe otro comando, \verb|\comp|, que hace
precisamente eso, permitir añadir pequeños comentarios directamente en
el propio párrafo (\texttt{comp} viene de \textbf{COM}entario en \textbf{P}árrafo).

% Igual que antes, no vale el "example".

El código:

\begin{verbatim}
El juego ``Vampire: the Masquerade'', publicado en 1998,
requirió 12 desarrolladores durante 24 meses, casi dos millones
de dólares y unas 366.000 líneas de código.\comp{300.000 para
el juego, y 66.000 de scripts.}
\end{verbatim}

Se convierte en:

\smallskip

El juego ``Vampire: the Masquerade'', publicado en 1998,
requirió 12 desarrolladores durante 24 meses, casi dos millones
de dólares y unas 366.000 líneas de código.\compImpl{300.000 para
el juego, y 66.000 de scripts.}

\subsection{Comando \texttt{todo}}
\label{cap3:subsec:comando-todo}

Este comando permite añadir comentarios para indicar tareas que aún
faltan por hacer. Los informáticos solemos marcar esos comentarios en
nuestro código fuente utilizando la ``palabra''
\texttt{TODO}\footnote{Que en realidad no tiene nada que ver con la
  palabra española, sino con las inglesas ``\emph{to do}'', que puede
traducirse aquí a ``\emph{por hacer}''.}.

El comando \verb+\todo+ encierra el texto entre llaves y lo
antecede con la marca ``TODO'' en negrita, de forma que el
código:

% Igual que antes, no vale el "example".

\begin{verbatim}
Existen autores que piensan que enseñar programación orientada
a objetos en el primer curso de programación (CS1) es
beneficioso para los alumnos\todo{Meter referencias...}.
\end{verbatim}

se convierte en la versión de depuración en:

\smallskip

Existen autores que piensan que enseñar programación orientada a
objetos en el primer curso de programación (CS1) es beneficioso para
los alumnos\todoImpl{Meter referencias...}.

\medskip

Y, al igual que los anteriores, cuando se compila el documento en
``modo release'', el comando no tiene ningún efecto.

\bigskip

Es importante destacar que en los dos comandos que van dentro de los
párrafos (\verb+\comp+ y \verb+\todo+) \emph{no
  se debe poner ningún espacio antes del comando}. En caso de ponerse
el espacio, éste \emph{aparecería} en la versión Release, cuando el
comando no tiene ningún efecto:

\com{Aquí utilizamos el entorno \texttt{example}, porque asumimos que
  la versión que estará leyendo el usuario es la Release. Si estás
  leyendo este texto, el ejemplo no lo entenderás, porque estás
  manejando la versión de depuración, y a la derecha verás también la
  sección ``\textbf{TODO}''.}

\begin{example}
... beneficioso para los
alumnos \todo{Meter 
referencias...}.
\end{example}

Para que cuando se genera el documento en modo depuración quede bien,
el propio comando \emph{añade} el espacio de separación entre el texto
que le precede y la apertura de la llave.

\medskip

Ten en cuenta, que al hacer uso de estos comandos para depuración (\verb|\com|, \verb|\comp| o \verb|\todo|) el documento generado contendrá más texto que el final en \emph{release}. Eso significa que el número de páginas variará, y la maquetación general también. Por tanto, \emph{no} debes utilizar el resultado de la generación en depuración para averiguar, por ejemplo, si una figura queda cerca del punto donde es referenciada, o si en una misma página aparecen dos elementos flotantes.

%-------------------------------------------------------------------
\section{Acelerando la compilación}
%-------------------------------------------------------------------
\label{cap3:sec:compilacion-rapida}

Cuando el documento va teniendo más y más páginas, compilarlo una y
otra vez hasta dar con el tamaño exacto que queremos darle a una
imagen, o para ver si una referencia queda bien generada a partir de
la entrada en el \texttt{.bib} puede llevar demasiado tiempo.

Para evitarlo, \texis\ permite, de manera fácil, compilar un único
capítulo (o apéndice), que normalmente será aquél en el que se esté trabajando.

Para eso, simplemente hay que indicar qué capítulo se quiere compilar
en el fichero \texttt{config.tex} utilizando el comando
\verb+\compilaCapitulo+\footnote{El comando sólo puede invocarse una
  vez, por lo que no es válido si se quiere compilar un grupo
  determinado de capítulos.}. Si en vez de ser un capítulo lo que
queremos generar es un apéndice el procedimiento es el mismo, pero
utilizando el comando \verb+\compilaApendice+. Observa que \emph{no}
debe incluirse el nombre del directorio donde aparecen los ficheros
(es decir el ``\texttt{Capitulos}''), pues el propio comando lo hace
por nosotros.

Una vez que el capítulo se termina de escribir y se pasa al siguiente,
se querrá añadir el \verb+\compilaCapitulo+ para el nuevo capítulo (y
anular el otro). En nuestro caso, en vez de eliminar el comando del
capítulo anterior, lo dejamos comentado por si es necesario en el
futuro. Es por ello que al final de la redacción del documento, se
tiene una línea por cada uno de los capítulos:

\begin{verbatim}
% Descomentar la línea para establecer el capítulo que queremos
% compilar

% \compilaCapitulo{01Introduccion}
% \compilaCapitulo{02EstructuraYGeneracion}
% \compilaCapitulo{03Edicion}
% \compilaCapitulo{04Imagenes}
% \compilaCapitulo{05Bibliografia}
% \compilaCapitulo{06Makefile}

% \compilaApendice{01AsiSeHizo}
\end{verbatim}

%-------------------------------------------------------------------
\section{Editores de \LaTeX\ y compilación}
%-------------------------------------------------------------------
\label{cap3:sec:editores}

Existen numerosas alternativas para editar los ficheros de \LaTeX
\citep[ver][sec. 2.3]{Flynn05}, y
si has escrito ya algún artículo, posiblemente ya tengas uno
``favorito''. Aunque el editor parezca poco importante (al fin y al
cabo lo importante es tu documento), en realidad pasarás mucho tiempo
utilizándolo, viendo sus colores, pulsando sus botones, y activando
sus teclas rápidas.

Evidentemente \texis\ no obliga a utilizar ningún editor en
concreto (faltaría más), aunque es posible que necesites hacer algunos
cambios en los ficheros para que se adecúen a lo que espera el
editor. Esto es especialmente cierto si pretendes generar el documento
final utilizando alguna opción del editor.

En la sección~\ref{cap2:sec:compilacion} mostrábamos cómo compilar
todos los \texttt{.tex} desde la línea de comandos. Sin embargo,
reconocemos que esto no es lo más cómodo\footnote{\texis\ tiene un
  fichero \texttt{Makefile} para la compilación en un único paso, que
  es explicado en el capítulo~\ref{cap:makefile}.}. Por lo tanto, si
el editor que tienes está preparado para \LaTeX\ (no utilizas el Bloc
de notas...), es muy posible que tenga algún botón o tecla rápida para
compilar el fichero abierto, ya sea con \texttt{latex} o
\texttt{pdflatex}.

Pues bien, en ese caso, debes comprobar cómo funciona exactamente el
editor, ya que muy posiblemente, el fichero que estarás editando
cuando quieras generar el documento no será el documento
\emph{maestro} (es decir, el que en la plantilla hemos llamado
\texttt{Tesis.tex}, y que contiene el punto de entrada e incluye todos
los demás). Por lo tanto, debes mirar de qué manera puedes hacer que
el fichero que se envía a \texttt{latex} sea el documento maestro.
Por ejemplo,
WinEdt\footnote{\url{http://www.winedt.com/}} permite crear
``proyectos'' donde se añaden ficheros y se especifica cuál es el
documento maestro; cuando se pulsa el botón de compilar,
independientemente del fichero activo en el editor, se manda compilar
el documento maestro.

Como se describe en la sección \ref{ap1:edicion}, nosotros utilizamos emacs \citep{emacsStallman} para crear nuestros ficheros \LaTeX. Como no podía ser de otro modo, \texis\ está preparado para integrarse con él, en particular
con el modo Auc\TeX\ que permite una edición cómoda de ficheros \TeX\ \citep{AtazLopezEmacs}.
En concreto, este modo dispone de una combinación de teclas para
lanzar la generación del documento final. En condiciones normales eso
implica enviar al programa \texttt{latex} el fichero que se está
editando; sin embargo, en nuestro caso lo normal es que el fichero
\emph{maestro} que hay que utilizar no es el que se está editando,
sino el fichero \texttt{Tesis.tex}. Para que funcione como queremos,
basta con añadir al final de los ficheros \texttt{tex} unas
indicaciones para que Auc\TeX\ utilice ese fichero como fichero
maestro:

% ¿¡Tendrá alguna repercusión que se meta en medio del fichero
% para el propio emacs!?
\begin{verbatim}
% Variable local para emacs, para  que encuentre el fichero
% maestro de compilación y funcionen mejor algunas teclas
% rápidas de AucTeX
%%%
%%% Local Variables:
%%% mode: latex
%%% TeX-master: "../Tesis.tex"
%%% End:
\end{verbatim}

Esta ``coletilla'' no es necesaria si utilizas cualquier otro editor.
Sin embargo \texis\ las tiene añadidas en todos los ficheros (y
también en los ficheros de los capítulos y apéndices de este manual).
Las líneas anteriores, además, son utilizadas por otras combinaciones
de teclas de Auc\TeX, como las que permiten navegar por todas las
secciones del documento.

%-------------------------------------------------------------------
\section{Control de versiones}
%-------------------------------------------------------------------
\label{cap3:sec:control-versiones}

Como veremos en el capítulo~\ref{cap:makefile}, el fichero
\texttt{Makefile} contiene algunos objetivos para realizar copias de
seguridad de todos los ficheros del documento.

Sin embargo en el mundo de desarrollo software es habitual utilizar
sistemas de control de versiones. Estos sistemas gestionan las
distintas versiones por las que van pasando los ficheros durante todo
el proceso de desarrollo. La necesidad de estas herramientas está
ampliamente reconocida, no sólo porque sirven como medio de copia de
seguridad que permite \emph{volver hacia atrás} ante algún fallo, sino
porque permite el trabajo simultáneo de dos o más
personas\footnote{Aunque esto en la redacción de una tesis no suele
  tener sentido, sí puede ser necesario en la elaboración de manuales,
  cuadernillos de prácticas u otros documentos para los que
  \texis\ puede utilizarse.}.

Existen varias alternativas para el control de versiones, tanto
comerciales como bajo licencia \ac{GPL}
o similares. El sistema por excelencia dentro del software libre
fue durante muchos años \ac{CVS}
\citep{CVS}, aunque hoy por hoy ha sido desbancado por Subversion
\citep{Subversion}.  Entre las herramientas comerciales, destacan
SourceSafe de
Microsoft\footnote{\url{http://msdn.microsoft.com/ssafe/}},
Perforce\footnote{\url{http://www.perforce.com/}} y
AccuRev\footnote{\url{http://www.accurev.com/}}.

Aunque es una decisión que los autores del documento tendrán que
tomar, aconsejamos el uso de uno de estos sistemas\footnote{En nuestro
  caso, utilizamos CVS para la escritura de las tesis, mientras que
  para la elaboración de la plantilla (y manual), utilizamos
  Subversion.}. Una vez que se tiene configurada la máquina servidora
que aloja el control de versiones (ver notas bibliográficas), se suben
los ficheros \emph{fuente} del documento, que pasarán a estar bajo el
control del servidor, lo que permitirá recuperar el estado del
documento en cualquier momento pasado (por lo que sirve también como
copia de seguridad).

Un punto importante es hacer que el sistema de control de versiones
\emph{ignore} los ficheros que son resultados de la generación del
fichero final (el PDF). Cuando se compila el documento, \LaTeX\ genera
numerosos ficheros temporales (con extensiones como \texttt{.aux} o
\texttt{.bbl}) que \emph{no} deben subirse al sistema control de
versiones. Cuando se utiliza CVS se elimina el problema creando en los
directorios un fichero de texto llamado \texttt{.cvsignore} que
contiene todos los ficheros que deben ser ignorados. A pesar de que en
la elaboración de la plantilla no utilizamos CVS, \texis\
incorpora esos ficheros para que puedan utilizarse en el proceso de
redacción de los documentos.

Si en vez de utilizar CVS estás utilizando Subversion, puedes hacer
que éste ignore los ficheros contenidos en el archivo
\texttt{.cvsignore} ejecutando la siguiente orden:

\begin{verbatim}
svn propset svn:ignore -F .cvsignore .
\end{verbatim}

\noindent en cada uno de los directorios que contengan el fichero. La
orden lo que hace es establecer la propiedad (\texttt{propset})
concreta para que el Subversion ignore (\texttt{svn:ignore}) los
ficheros que se indican en el fichero (\texttt{-F})
\texttt{.cvsignore}.

\com{
Para una versión futura de \texis, podríamos incluir información
sobre la  revisión de la
plantilla en la página posterior a la portada, igual que en el libro
del SVN. En ese caso, habría que contarlo aquí para que los usuarios
puedan utilizarlo también.
}

%-------------------------------------------------------------------
\section*{\NotasBibliograficas}
%-------------------------------------------------------------------
\TocNotasBibliograficas

La idea de los dos modos de compilación de la Tesis surgió de forma
natural dada la experiencia en el proceso de desarrollo en C++, donde
los entornos integrados de desarrollo suelen proporcionar al menos
esas dos configuraciones posibles. La forma de hacerlo posible vino
después de inspeccionar el código \LaTeX\ del libro
\cite{ldesc2e}. La implementación de los comandos no requiere un
conocimiento ni mucho menos extenso de las capacidades de \TeX; basta
con un poco de intuición al ver un ejemplo de \verb|\if|.

No obstante, el lector interesado en aprender \TeX\ a fondo puede
encontrar diversos manuales, como ``\TeX\ for the Impatient''
\citep{texImpatient}, aunque advertimos que se debe estar \emph{muy}
interesado para leerselo, ya que en condiciones normales no se
utilizará nada de lo aprendido\footnote{A no ser que se quiera
  construir un paquete con una funcionalidad muy concreta...}. También
se puede consultar \cite{texKnuth} o \cite{texByTopic}.

Con respecto a la utilización de control de versiones, dentro de las
opciones libres es muy utilizado el Subversion, cuyo libro de
referencia que ya se ha citado en el texto es
\cite{Subversion}. Para una descripción sencilla de cómo instalar una
máquina servidora puede consultarse \citet{LatexAndSVN} y
\citet{PaquetesSVNLatex}. En éste último también aparece una somera
descripción de algunos paquetes de \LaTeX\ que pueden utilizarse para
incluir información relacionada directamente con las versiones de
Subversion. Aunque para más información al respecto recomendamos
\citet{SVNmulti} que dedica toda su atención a la descripción de
\verb+svn-multi+, uno de los paquetes con más opciones disponibles
para ello.

%-------------------------------------------------------------------
\section*{\ProximoCapitulo}
%-------------------------------------------------------------------
\TocProximoCapitulo

En este capítulo hemos tratado los aspectos más importantes desde el
punto de vista de la edición de un documento realizado con \texis,
describiendo los comandos \LaTeX\ disponibles.

El próximo capítulo aborda el tratamiento de las imágenes. Como se verá,
soportar la generación del documento
tanto con \texttt{latex} como \texttt{pdflatex} dificulta la gestión
de imágenes, pues cada uno utiliza un formato de fichero distinto. El
capítulo explica las distintas opciones que el usuario de \texis\
tiene para su manejo.

% Variable local para emacs, para  que encuentre el fichero maestro de
% compilación y funcionen mejor algunas teclas rápidas de AucTeX
%%%
%%% Local Variables:
%%% mode: latex
%%% TeX-master: "../ManualTeXiS.tex"
%%% End:
