\documentclass{article}

\usepackage[utf8x]{inputenc}
\usepackage[margin=2cm]{geometry}
\usepackage{verbatim}
\usepackage{graphicx}
\usepackage[spanish]{babel}
\usepackage{amsfonts, amssymb}
\usepackage{amsmath}
\usepackage{tikz-qtree}
\usepackage{mathtools}
\usepackage{fancybox}
\usetikzlibrary{positioning}
\usepackage{multicol}
\begin{document}

\title{Estructuras Discretas 2015-1\\
Facultad de Ciencias\\
Tarea 01 \\ \large {Parte 01}}
\author{Fernanda Sánchez \quad Najla Ochoa}
\date{Fecha de entrega: \\ 1 de septiembre de 2014}
\maketitle

\section*{Gramáticas y árboles de derivación}
\begin{enumerate}

	\item{Usando la siguiente gramática de expresiones aritméticas, da la secuencia de producciones que utilizas para generar la expresión $(-b + (b  \cdot b - 4  \cdot a  \cdot c)) \div (2  \cdot a)$. Dibuja el árbol que le corresponde a esa expresión.}\\
		$(a)$ $S ::= E $\\
		$(b)$ $E ::= var $\\
        	$(c)$ $E ::= const $\\			
		$(d)$ $E ::= \rhd E $\\
		$(e)$ $E ::= E \diamond E$\\
		$(f)$ $E ::= (E)$\\
		$(g)$ $var ::= a|b|...$\\
		$(k)$ $const ::= 0|1|2|3|4|...$\\
		$(i)$ $\rhd ::= +|- $\\
		$(j)$ $\diamond ::= +|-|\cdot|\div$\\ \\
		
		\large{\textbf{Respuesta:}} \normalsize \\

		\begin{multicols}{2}
				$1. S$\\
				$2. E$ \\
				$3. E \diamond E$ \\
				$4. (E) \diamond E$ \\
				$5. (E) \diamond (E)$ \\
				$6. (E \diamond E) \diamond (E)$ \\
				$7. (E \diamond (E)) \diamond (E)$ \\  	
				$8. (\rhd E \diamond (E)) \diamond (E)$\\
				$9. (- E \diamond (E)) \diamond (E)$\\
				$10. (- var \diamond (E)) \diamond (E)$\\
				$11. (- b \diamond (E)) \diamond (E)$\\
				$12. (- b + (E)) \diamond (E)$ \\	
				$13. (- b + (E \diamond E)) \diamond (E)$ \\ 	
				$14. (- b + (E - E)) \diamond (E)$ \\
				$15. (- b + (E \diamond E - E)) \diamond (E)$ \\
				$16. (- b + (E \diamond E - E \diamond E)) \diamond (E)$ \\
				$16. (- b + (E \diamond E - E \diamond E \diamond E)) \diamond (E)$\\ 	
				$18. (- b + (E \cdot E - E \diamond E \diamond E)) \diamond (E)$ \\
				$19. (- b + (var \cdot E - E \diamond E \diamond E)) \diamond (E)$ \\ 
				$20. (- b + (b \cdot E - E \diamond E \diamond E)) \diamond (E)$  \\
				$21. (- b + (b \cdot var - E \diamond E \diamond E)) \diamond (E)$ \\
				$22. (- b + (b \cdot b - E \diamond E \diamond E)) \diamond (E)$ \\
				$23. (- b + (b \cdot b - const \diamond E \diamond E)) \diamond (E)$ \\
				$24. (- b + (b \cdot b - 4 \diamond E \diamond E)) \diamond (E)$ \\
				$25. (- b + (b \cdot b - 4 \cdot E \diamond E)) \diamond (E)$ \\
				$26. (- b + (b \cdot b - 4 \cdot var \diamond E)) \diamond (E)$ \\
				$27. (- b + (b \cdot b - 4 \cdot a \diamond E)) \diamond (E)$ \\
				$28. (- b + (b \cdot b - 4 \cdot a \cdot E)) \diamond (E)$ \\
				$29. (- b + (b \cdot b - 4 \cdot a \cdot var)) \diamond (E)$ \\
				$30. (- b + (b \cdot b - 4 \cdot a \cdot c)) \diamond (E)$ \\
				$31. (- b + (b \cdot b - 4 \cdot a \cdot c)) \div (E)$ \\
				$32. (- b + (b \cdot b - 4 \cdot a \cdot c)) \div (E \diamond E)$ \\
				$33. (- b + (b \cdot b - 4 \cdot a \cdot c)) \div (E \cdot E)$ \\
				$34. (- b + (b \cdot b - 4 \cdot a \cdot c)) \div (const \cdot E)$ \\
				$35. (- b + (b \cdot b - 4 \cdot a \cdot c)) \div (2 \cdot E)$ \\
				$36. (- b + (b \cdot b - 4 \cdot a \cdot c)) \div (2 \cdot var)$ \\
				$37. (- b + (b \cdot b - 4 \cdot a \cdot c)) \div (2 \cdot a)$ \\
			\end{multicols}

	
			\begin{tikzpicture}
			\Tree [.\node {S};
				[.\node {E};
					[.\node {E};
						[.\node {$($};]
						[.\node {E};
							[.\node {E};
								[.\node {$\rhd$};
									[.\node {$-$};]]
								[.\node {E};
									[.\node {var};
										[.\node {b};]]]]
							[.\node {$\diamond$};
								[.\node {$+$};]]
							[.\node {E};
								[.\node {$($};]
								[.\node {E};
									[.\node {E};
										[.\node {E};
											[.\node {var};
											[.\node {b};]]]
                                                                        	[.\node {$\diamond$};
											[.\node {$\cdot$};]]
                                                                        	[.\node {E};
											[.\node {var};
                                                                                        [.\node {b};]]]]
									[.\node {$\diamond$};
										[.\node {$-$};]]
									[.\node {E};
										[.\node {E};
											[.\node {const};
                                                                                        [.\node {4};]]]
                                                                        	[.\node {$\diamond$};
											[.\node {$\cdot$};]]
                                                                        	[.\node {E};
											[.\node {E};
												[.\node {var};
                                                                                        	[.\node {a};]]]
                                                                               		[.\node {$\diamond$};
												[.\node {$\cdot$};]]
                                                                                	[.\node {E};
												[.\node {var};
                                                                                        	[.\node {c};]]]]]]
								[.\node {$)$};]]]
						[.\node {$)$};]]
					[.\node {$\diamond$};
						[.\node {$\div$};]]
					[.\node {E};
 						[.\node {$($};]
                                                [.\node {E};
							[.\node {E};
								[.\node {const};
								[.\node {2};]]]
							[.\node {$\diamond$};
								[.\node {$\cdot$};]]
							[.\node {E};
								[.\node {var};
								[.\node {a};]]]]
                                                [.\node {$)$};]]]]
			\end{tikzpicture}
					
		
	\item{Dado el siguiente árbol, escribe una posible gramática asociada a éste y las producciones que necesitas para llegar a la expresión que genera.}

	
\begin{center}	
\begin{tikzpicture}
\Tree [.\node(a) {C}; 
          [.\node (b) {1}; ] 
          [.\node (c) {C};
		[.\node(e) {4};]
		[.\node(f) {C};
			[.\node(h) {5};]
			[.\node(i) {C};
				[.\node(k) {a};]
				[.\node(l) {C};
					[.\node(n) {+};]]
				[.\node(m) {a};]]
			[.\node(j) {5};]]
		[.\node(g) {4};] ] 
          [.\node (d) {1}; ]
      ]

\end{tikzpicture}	
\end{center}

\large{\textbf{Respuesta:}} \normalsize \\

Una posible gramática asociada con el árbol de la pregunta es: \\
	$(1)$ $C ::= 1C1$ \\
	$(2)$ $C ::= 4C4$ \\
	$(3)$ $C ::= 5C5$ \\
	$(4)$ $C ::= aCa$ \\
	$(5)$ $C ::= +$ \\
\\ A partir de esta gramática para generar la expresión $145a+a541$ se utilizan las siguientes producciones: \\
	1. $C$ \\
	2. $1C1$ \\ 
	3. $14C41$ \\
	4. $145C541$ \\
	5. $145aCa541$ \\
	6. $145a+a541$ \\


\section*{El lenguaje de la lógica proposicional}

	\item{¿Cuáles de las siguientes oraciones son proposiciones atómicas, cuáles no atómicas y cuáles no son proposiciones?}
		\\ \textbf{\large {Respuestas: }} \normalsize		
		\begin{itemize}
			\item{El perro come papel:} \textbf{Proposición atómica}
			\item{Gracias:} \textbf{No es proposición}
			\item{$(x + 5) + \sqrt{i} = 5 $:} \textbf{Proposición atómica}
			\item{$(x + 5) + \sqrt{i}$:} \textbf{No es proposición}
			\item{Pablo y su amiga van al parque:} \textbf{Proposición no atómica}
			\item{$x \leq 0 \leq y$:} \textbf{Proposición no atómica}
		\end{itemize}
	\item{Usando las variables proposicionales $l$ y $a$ para denotar a las proposiciones atómicas \emph{Juan es muy listo} y \emph{Juan está satisfecho} respectivamente, denota con estas variables proposicionales y los conectivos lógicos a las siguientes proposiciones lógicas.}
		\\ \textbf{\large {Respuestas: }} \normalsize	
		\begin{itemize}
			\item{Juan es muy listo y está satisfecho.} \textbf{ Fórmula: $l \wedge a$}
			\item{Si Juan no fuera listo, no estaría satisfecho.} \textbf{ Fórmula: $\neg l \rightarrow \neg a$}
			\item{Juan está satisfecho únicamente si es listo.} \textbf{Fórmula: $a \leftrightarrow l$}
			\item{Juan es listo pero no está satisfecho.} \textbf{Fórmmula: $l \wedge \neg a$}
		\end{itemize}
	\item{Para cada uno de los siguientes enunciados, asigna variables proposicionales y escribe la fórmula o argumento lógico correspondiente al enunciado.}
		\begin{itemize}
			\item Si Pepito compite en natación va a ganar el primer lugar. Si Juanito compite en natación va a ganar el primer lugar. Alguno de los dos no va a quedar en primer lugar en la competencia de natación. Por lo tanto o Pepito no compite o Juanito no compite.\\
		
			\textbf{\large Respuesta:}\normalsize 
			Pepito compite en natación: $p$ \\
			Juanito compite en natación: $j$ \\
			Ganar el 1er lugar: $l$	\\

			1. $p \rightarrow l$\\
			2. $j \rightarrow l$\\
			3. $\neg (p \wedge l) \vee \neg (j \wedge l)$\\
			\noindent\rule{3cm}{0.4pt} \\
			$\therefore \neg p \vee \neg j$ \\
			\item Voy a comer tacos o quesadillas. Decidí no comer quesadillas. Entonces comeré tacos. \\
			\\ \textbf{\large Respuesta:\\}\normalsize \\
			Comer tacos: $t$ \\
			Comer quesadillas: $q$ \\ 			
			1. $t \vee q$ \\
			2. $\neg q$ \\
			\noindent\rule{3cm}{0.4pt} \\
			$\therefore t$
			
		\end{itemize}
	\item{Construye la tabla de verdad para cada una de las siguientes fórmulas, clasificando si se trata de una tautología, contadicción o contingencia.}
		
		\textbf{\large Respuestas:}\normalsize
		\begin{itemize}
			\item{$(p \wedge (r \wedge q)) \rightarrow r$}: \textbf{Contingencia}
			\item{$((p \rightarrow q) \wedge \neg q) \rightarrow p$}: \textbf{Contingencia}
			\item{$(q \vee p) \rightarrow (\neg p \rightarrow q)$}: \textbf{Tautología}
		\end{itemize}
	\item{Analiza mediante tablas de verdad la correctud de los siguientes argumentos.}
		\\ \textbf{\large Respuestas:}\normalsize 
		\begin{itemize}
			\item{$p,q / \therefore p \wedge q $}: \textbf{El argumento es correcto.}
			\item{$p \rightarrow q, \neg q / \therefore \neg p$}: \textbf{El argumento es correcto.}
			\item{$p \rightarrow q \vee r, \neg q / \therefore p \rightarrow r$}: \textbf{El argumento no es correcto.}
		\end{itemize}	
\section*{Evaluación de expresiones}
	\item{Coloca los paréntesis en las siguientes expresiones de acuerdo a la precedencia y asociatividad de los operadores.}
		\textbf{\large Respuestas:}\normalsize    
		\begin{itemize}
			\item{$(((-b) + (b ** 2)) - ((((4 \cdot a) \cdot c) / 2) \cdot a))$}
			\item{$((((p \wedge q) \vee r) \rightarrow s) \leftrightarrow (a \vee q))$}
			\item{$((p \wedge q) \rightarrow ((\neg r) \vee q))$}
		\end{itemize}
	\item{Realiza las siguientes sustituciones textuales.}
		\begin{itemize}
			\item{$x + y \cdot x [x \coloneqq y + 2][y \coloneqq y \cdot x] $}\\
			\textbf{Sustitución:} $x + y \cdot (y \cdot x +2)$\\
			\item{$(x + y \cdot x) [x \coloneqq y + 2][y \coloneqq y \cdot x] $}\\
			\textbf{Sustitución:} $y \cdot x + 2 + y \cdot x \cdot y \cdot x + 2$\\
			\item{$(x + x \cdot 2)[x,y \coloneqq x, z][x \coloneqq y] $}\\
			\textbf{Sustitución:} $y + y \cdot 2 $\\
			\item{$(x + x \cdot y + x \cdot y \cdot z)[x,y \coloneqq y,x][y \coloneqq 2 \cdot y]$}\\	
			\textbf{Sustitución:} $2 \cdot y + 2 \cdot y \cdot x + 2 \cdot y \cdot x \cdot z $
		\end{itemize}
\section*{Análisis sintáctico de expresiones lógicas}
	\item{De los siguientes enunciados, define el conectivo principal y con base en éste obtén el esquema básico. Para cada operador: si el operador es binario especifica su rango izquierdo y su rango derecho; si el operador es unario, especifica su rango(derecho). Finalmente, construye su árbol de análisis sintáctico.}
		\begin{itemize}
			\item {$p \vee (\neg p \wedge q) \rightarrow p \vee q$}
			\item {$\neg (p \wedge q \rightarrow p \vee q)$}
			\item {$\neg p \vee q \rightarrow p \wedge \neg q \rightarrow \neg p \vee \neg q \rightarrow \neg p \wedge \neg q$}

		\end{itemize}
\end{enumerate}

\textbf{Extra:}¿Qué es sintaxis?¿Qué es semántica?

\end{document}
