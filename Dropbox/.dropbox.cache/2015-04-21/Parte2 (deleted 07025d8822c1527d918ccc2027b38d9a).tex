%---------------------------------------------------------------------
%
%                          Parte 2
%
%---------------------------------------------------------------------
%
% Parte2.tex
% Copyright 2009 Marco Antonio Gomez-Martin, Pedro Pablo Gomez-Martin
%
% This file belongs to the TeXiS manual, a LaTeX template for writting
% Thesis and other documents. The complete last TeXiS package can
% be obtained from http://gaia.fdi.ucm.es/projects/texis/
%
% Although the TeXiS template itself is distributed under the 
% conditions of the LaTeX Project Public License
% (http://www.latex-project.org/lppl.txt), the manual content
% uses the CC-BY-SA license that stays that you are free:
%
%    - to share & to copy, distribute and transmit the work
%    - to remix and to adapt the work
%
% under the following conditions:
%
%    - Attribution: you must attribute the work in the manner
%      specified by the author or licensor (but not in any way that
%      suggests that they endorse you or your use of the work).
%    - Share Alike: if you alter, transform, or build upon this
%      work, you may distribute the resulting work only under the
%      same, similar or a compatible license.
%
% The complete license is available in
% http://creativecommons.org/licenses/by-sa/3.0/legalcode
%
%---------------------------------------------------------------------

% Definici�n de la segunda parte del manual

\partTitle{Conceptos avanzados}

\partDesc{Esta segunda parte del manual contiene cap�tulos que pueden
considerarse ``avanzados'', aunque cualquier documento a buen seguro
har� uso de los conceptos que en ellos se presentan.

Un primer cap�tulo explica la gesti�n de las im�genes que \texis\
espera que se utilice. El manual pasa despu�s a explicar c�mo a�adir
bibliograf�a y acr�nimos. Por �ltimo, se describe el fichero {\tt
Makefile} proporcionado, que ayuda en algunas de las tareas de
generaci�n de documentos.}

\makepart
