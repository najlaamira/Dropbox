%---------------------------------------------------------------------
%
%                          TeXiS_toc.tex
%
%---------------------------------------------------------------------
%
% TeXiS_toc.tex
% Copyright 2009 Marco Antonio Gomez-Martin, Pedro Pablo Gomez-Martin
%
% This file belongs to TeXiS, a LaTeX template for writting
% Thesis and other documents. The complete last TeXiS package can
% be obtained from http://gaia.fdi.ucm.es/projects/texis/
%
% This work may be distributed and/or modified under the
% conditions of the LaTeX Project Public License, either version 1.3
% of this license or (at your option) any later version.
% The latest version of this license is in
%   http://www.latex-project.org/lppl.txt
% and version 1.3 or later is part of all distributions of LaTeX
% version 2005/12/01 or later.
%
% This work has the LPPL maintenance status `maintained'.
% 
% The Current Maintainers of this work are Marco Antonio Gomez-Martin
% and Pedro Pablo Gomez-Martin
%
%---------------------------------------------------------------------
%
% Contiene  los  comandos  para  generar los  �ndices  del  documento,
% entendiendo por �ndices las tablas de contenidos.
%
% Genera  el  �ndice normal  ("tabla  de  contenidos"),  el �ndice  de
% figuras y el de tablas. Tambi�n  crea "marcadores" en el caso de que
% se est� compilando con pdflatex para que aparezcan en el PDF.
%
%---------------------------------------------------------------------


% Primero un poquito de configuraci�n...


% Pedimos que inserte todos los ep�grafes hasta el nivel \subsection en
% la tabla de contenidos.
\setcounter{tocdepth}{2} 

% Le  pedimos  que nos  numere  todos  los  ep�grafes hasta  el  nivel
% \subsubsection en el cuerpo del documento.
\setcounter{secnumdepth}{3} 


% Creamos los diferentes �ndices.

% Lo primero un  poco de trabajo en los marcadores  del PDF. No quiero
% que  salga una  entrada  por cada  �ndice  a nivel  0...  si no  que
% aparezca un marcador "�ndices", que  tenga dentro los otros tipos de
% �ndices.  Total, que creamos el marcador "�ndices".
% Antes de  la creaci�n  de los �ndices,  se a�aden los  marcadores de
% nivel 1.

\ifpdf
   \pdfbookmark{�ndices}{indices}
\fi

% Tabla de contenidos.
%
% La  inclusi�n  de '\tableofcontents'  significa  que  en la  primera
% pasada  de  LaTeX  se  crea   un  fichero  con  extensi�n  .toc  con
% informaci�n sobre la tabla de contenidos (es conceptualmente similar
% al  .bbl de  BibTeX, creo).  En la  segunda ejecuci�n  de  LaTeX ese
% documento se utiliza para  generar la verdadera p�gina de contenidos
% usando la  informaci�n sobre los  cap�tulos y dem�s guardadas  en el
% .toc
\ifpdf
   \pdfbookmark[1]{Tabla de contenidos}{tabla de contenidos}
\fi

\cabeceraEspecial{\'Indice}

\tableofcontents

\newpage 

% �ndice de figuras
%
% La idea es semejante que para  el .toc del �ndice, pero ahora se usa
% extensi�n .lof (List Of Figures) con la informaci�n de las figuras.

\cabeceraEspecial{\'Indice de figuras}

\ifpdf
   \pdfbookmark[1]{�ndice de figuras}{indice de figuras}
\fi

\listoffigures

\newpage

% �ndice de tablas
% Como antes, pero ahora .lot (List Of Tables)

\ifpdf
   \pdfbookmark[1]{�ndice de tablas}{indice de tablas}
\fi

\cabeceraEspecial{\'Indice de tablas}

\listoftables

\newpage

% Variable local para emacs, para  que encuentre el fichero maestro de
% compilaci�n y funcionen mejor algunas teclas r�pidas de AucTeX

%%%
%%% Local Variables:
%%% mode: latex
%%% TeX-master: "../Tesis.tex"
%%% End:
